% Copyright 2019 by Till Tantau
% Copyright 2019 by Jannis Pohlmann
%
% This file may be distributed and/or modified
%
% 1. under the LaTeX Project Public License and/or
% 2. under the GNU Public License.
%
% See the file doc/generic/pgf/licenses/LICENSE for more details.

\ProvidesFileRCS{tikzlibrarygraphs.code.tex}


%
% Interface keys
%

\def\tikzgraphsset{\pgfqkeys{/tikz/graphs}}%

\tikzgraphsset{
  new ->/.code n args={4}{%
    \path [->,every new ->/.try]
      (#1\tikzgraphleftanchor)
      edge[#3] #4
      (#2\tikzgraphrightanchor);},
  new --/.code n args={4}{
      \path [-,every new --/.try]
      (#1\tikzgraphleftanchor)
      edge[#3] #4
      (#2\tikzgraphrightanchor);},
  new <->/.code n args={4}{
    \path [<->,every new <->/.try]
      (#1\tikzgraphleftanchor)
      edge[#3] #4
      (#2\tikzgraphrightanchor);},
  new -!-/.code n args={4}{},
  new <-/.code n args={4}{%
    \path [<-,every new <-/.try]
      (#1\tikzgraphleftanchor)
      edge[#3] #4
      (#2\tikzgraphrightanchor);
  }
}%


\def\tikz@lib@graph@store@anchor#1#2{%
  \def\tikz@temp{#1}%
  \ifx\tikz@temp\pgfutil@empty%
    \let#2\tikz@temp%
  \else%
    \def\tikz@temp{.#1}%
    \let#2\tikz@temp%
  \fi%
}%

\tikzgraphsset{
  default edge kind/.initial=--,
  --/.style={default edge kind=--},
  ->/.style={default edge kind=->},
  <-/.style={default edge kind=<-},
  <->/.style={default edge kind=<->},
  -!-/.style={default edge kind=-!-},
  default edge operator/.initial=matching and star,
  left anchor/.code=\tikz@lib@graph@store@anchor{#1}{\tikzgraphleftanchor},
  right anchor/.code=\tikz@lib@graph@store@anchor{#1}{\tikzgraphrightanchor},
  left anchor=,
  right anchor=
}%


%
% Keys for using nodes declared outside a graph inside a graph as if
% it were declared there
%

\tikzgraphsset{
  use existing nodes/.is if=tikz@lib@graph@all
}%

\tikzset{
  new set/.code={
    \expandafter\global\expandafter\let\csname tikz@lg@node@set #1\endcsname\pgfutil@empty%
  },
  set/.code={
    \tikz@fig@mustbenamed%
    \ifcsname tikz@lg@node@set #1\endcsname\else
      \tikzerror{Undefined set `#1'}%
    \fi
    \expandafter\def\expandafter\tikz@alias\expandafter{\tikz@alias%
      \expandafter\def\expandafter\pgf@temp\expandafter{\csname tikz@lg@node@set #1\endcsname}%
      \expandafter\expandafter\expandafter\pgfutil@g@addto@macro\expandafter\pgf@temp\expandafter{\expandafter\tikz@lg@do\expandafter{\tikz@fig@name}}%
    }%
  },%
}%

\newif\iftikz@lib@graph@all


%
% Simple versus multi graphs
%
\tikzgraphsset{
  simple/.code={
    \iftikz@lib@graph@simple%
      % is already simple, ignore
    \else
      \tikz@lib@graph@simpletrue%
      \pgfkeysalso{operator=\tikz@lib@graph@simple@done}%
    \fi%
  },
  multi/.code={
    \tikz@lib@graph@simplefalse%
  }
}%

\newif\iftikz@lib@graph@simple

\def\tikz@lib@graph@set@simple@edge#1#2#3#4#5{%
  % #1 = kind (<-, ->, ...)
  % #2 = from
  % #3 = to
  % #4 = options
  % #5 = edge nodes
  %
  % Ok, first, test, whether edge exists:
  \ifcsname tikz@lg@e@#3@#2\endcsname%
    \expandafter\global\expandafter\let\csname tikz@lg@e@#3@#2\endcsname\relax% reset
  \fi%
  \expandafter\gdef\csname tikz@lg@e@#2@#3\endcsname{\tikz@lib@graph@make@simple@edge{#1}{#2}{#3}{#4}{#5}}%
}%

\def\tikz@lib@graph@make@simple@edge#1#2#3#4#5{%
  \pgfqkeys{/tikz/graphs}{new #1={#2}{#3}{#4}{#5}}%
}%


\def\tikz@lib@graph@simple@done{%
  \tikz@lib@graph@pack@node@list%
  {%
    \let\tikz@lg@do\tikz@lib@graph@simple@node%
    \tikz@lib@graph@node@list
  }%
}%

\def\tikz@lib@graph@simple@node#1{%
  {%
    \def\tikz@lib@graph@simple@from@node{#1}%
    \let\tikz@lg@do\tikz@lib@graph@simple@other@node%
    \tikz@lib@graph@node@list%
  }%
}%

\def\tikz@lib@graph@simple@other@node#1{%
  \ifcsname tikz@lg@e@\tikz@lib@graph@simple@from@node @#1\endcsname%
    \csname tikz@lg@e@\tikz@lib@graph@simple@from@node @#1\endcsname%
    \expandafter\global\expandafter\let\csname tikz@lg@e@\tikz@lib@graph@simple@from@node @#1\endcsname\relax%
  \fi%
}%


%
% Basic options
%

\tikzgraphsset{
  @nodes styling/.style=,
  nodes/.style={/tikz/graphs/@nodes styling/.append style={,#1}},
  @edges styling/.initial=,
  edges/.style={/tikz/graphs/@edges styling/.append={,#1}},
  edge/.style={edges={#1}},
  @edges node/.initial=,
  edge node/.style={/tikz/graphs/@edges node/.append={#1}},
  edge label/.style={/tikz/graphs/@edges node/.append={node[auto]{#1}}},
  edge label'/.style={/tikz/graphs/@edges node/.append={node[auto,swap]{#1}}},
  @operators/.initial=,
  operator/.style={/tikz/graphs/@operators/.append={#1}},
  @extra group options/.style=,
}%


\def\tikzgraphinvokeoperator#1{%
  {%
    \pgfkeyslet{/tikz/graphs/@operators}\pgfutil@empty%
    \pgfkeys{/tikz/graphs/.cd,#1}%
    \pgfkeysgetvalue{/tikz/graphs/@operators}\tikz@lib@graph@temp%
    \global\let\tikz@lib@graph@temp\tikz@lib@graph@temp%
  }%
  \tikz@lib@graph@temp%
  \global\let\tikz@lib@graph@temp\relax%
}%


%
% The parser
%

\def\tikz@lib@graph@parser{%
  \pgfutil@ifnextchar[{\tikz@lib@graph@parser@}{\tikz@lib@graph@parser@[]}%]
}%

\def\tikz@lib@graph@parser@[#1]{%
  \setbox\tikz@whichbox=\hbox\bgroup%
    \unhbox\tikz@whichbox%
    \hbox\bgroup
      \bgroup%
        \pgfinterruptpath%
          \scope[graphs/.cd,@graph drawing setup/.try,@operators=,every graph/.try,#1]%
            \begingroup%
            \iftikz@graph@quick%
              \expandafter\tikz@lib@graphs@parse@quick@graph%
            \else%
              \expandafter\tikz@lib@graphs@normal@main%
            \fi%
}%

\long\def\tikz@lib@graphs@normal@main#1{%
              \pgfkeysgetvalue{/tikz/graphs/@operators}\tikz@lib@graph@outer@operators%
              \let\tikz@lib@graph@options\pgfutil@empty%
              \let\tikz@lib@graph@node@list\pgfutil@empty%
              \pgfkeyssetvalue{/tikz/graphs/placement/depth}{0}%
              \pgfkeyssetvalue{/tikz/graphs/placement/width}{0}%
              \pgfkeyssetvalue{/tikz/graphs/placement/level}{0}%
              \tikz@lib@graph@start@hint@group%
                \tikz@lib@graph@parse@group{#1}%
              \tikz@lib@graph@end@hint@group
              \tikz@lib@graph@outer@operators%
              \let\tikz@lg@do=\tikz@lib@graph@cleanup%
              \tikz@lib@graph@node@list%
              \tikz@lib@graph@main@done%
}%

\def\tikz@lib@graph@main@done{%
            \endgroup%
          \endscope%
        \endpgfinterruptpath%
      \egroup
    \egroup%
  \egroup%
  \tikz@lib@graph@parser@done%
}%





\def\tikz@lib@graph@start@hint@group{%
  \pgfkeyssetvalue{/tikz/graphs/placement/local depth}{0}%
  \pgfkeyssetvalue{/tikz/graphs/placement/local width}{0}%
  \pgfkeyssetvalue{/tikz/graphs/placement/chain count}{0}%
  \pgfkeyssetvalue{/tikz/graphs/placement/element count}{0}%
}%

\def\tikz@lib@graph@end@hint@group{%
  % Get local depth and width outside
  \xdef\tikz@lib@graph@group@depth{\pgfkeysvalueof{/tikz/graphs/placement/local depth}}
  \xdef\tikz@lib@graph@group@width{\pgfkeysvalueof{/tikz/graphs/placement/local width}}
}%

\def\tikz@lib@graph@hint@aftergroup{%
  \pgfkeysgetvalue{/tikz/graphs/placement/width}\tikz@temp@h%
  \pgfkeysgetvalue{/tikz/graphs/placement/local width}\tikz@temp@lh%
  \pgfkeysgetvalue{/tikz/graphs/placement/local depth}\tikz@temp@lw%
  \pgfmathsetmacro\tikz@temp@h{\tikz@lib@graph@group@width+\tikz@temp@h}
  \pgfmathsetmacro\tikz@temp@lh{\tikz@lib@graph@group@width+\tikz@temp@lh}
  \pgfmathsetmacro\tikz@temp@lw{max(\tikz@lib@graph@group@depth,\tikz@temp@lw)}
  \pgfkeyslet{/tikz/graphs/placement/width}\tikz@temp@h%
  \pgfkeyslet{/tikz/graphs/placement/local width}\tikz@temp@lh%
  \pgfkeyslet{/tikz/graphs/placement/local depth}\tikz@temp@lw%
  %
  \pgfkeysgetvalue{/tikz/graphs/placement/element count}\tikz@temp%
  \c@pgf@counta=\tikz@temp\relax%
  \advance\c@pgf@counta by1\relax%
  \edef\tikz@temp{\the\c@pgf@counta}%
  \pgfkeyslet{/tikz/graphs/placement/element count}\tikz@temp%
}%

\def\tikz@lib@graph@placement@update{%
  \pgfkeys{/tikz/graphs/placement/logical node depth/.expand once=\tikz@lib@graph@name}
  \let\tikz@lib@graph@node@depth\pgfmathresult
  \pgfkeys{/tikz/graphs/placement/logical node width/.expand once=\tikz@lib@graph@name}
  \let\tikz@lib@graph@node@width\pgfmathresult
  % This is a single node, so update the lengths accordingly
  \pgfkeysgetvalue{/tikz/graphs/placement/width}\tikz@temp@h%
  \pgfkeysgetvalue{/tikz/graphs/placement/local width}\tikz@temp@lh%
  \pgfkeysgetvalue{/tikz/graphs/placement/local depth}\tikz@temp@lw%
  \pgfmathsetmacro\tikz@temp@h{\tikz@lib@graph@node@width+\tikz@temp@h}
  \pgfmathsetmacro\tikz@temp@lh{\tikz@lib@graph@node@width+\tikz@temp@lh}
  \pgfmathsetmacro\tikz@temp@lw{max(\tikz@lib@graph@node@depth,\tikz@temp@lw)}
  \pgfkeyslet{/tikz/graphs/placement/width}\tikz@temp@h%
  \pgfkeyslet{/tikz/graphs/placement/local width}\tikz@temp@lh%
  \pgfkeyslet{/tikz/graphs/placement/local depth}\tikz@temp@lw%
  %
  \pgfkeysgetvalue{/tikz/graphs/placement/element count}\tikz@temp%
  \c@pgf@counta=\tikz@temp\relax%
  \advance\c@pgf@counta by1\relax%
  \edef\tikz@temp{\the\c@pgf@counta}%
  \pgfkeyslet{/tikz/graphs/placement/element count}\tikz@temp%
}%

\def\tikz@lib@graph@placement@after@chain@update{%
  \pgfkeysgetvalue{/tikz/graphs/placement/depth}\tikz@temp@w%
  \pgfkeysgetvalue{/tikz/graphs/placement/local width}\tikz@temp@lh%
  \pgfkeysgetvalue{/tikz/graphs/placement/local depth}\tikz@temp@lw%
  \pgfmathsetmacro\tikz@temp@w{\tikz@lib@graph@chain@depth+\tikz@temp@w}%
  \pgfmathsetmacro\tikz@temp@lw{\tikz@lib@graph@chain@depth+\tikz@temp@lw}%
  \pgfmathsetmacro\tikz@temp@lh{max(\tikz@lib@graph@chain@width,\tikz@temp@lh)}%
  \pgfkeyslet{/tikz/graphs/placement/depth}\tikz@temp@w%
  \pgfkeyslet{/tikz/graphs/placement/local width}\tikz@temp@lh%
  \pgfkeyslet{/tikz/graphs/placement/local depth}\tikz@temp@lw%
  %
  \pgfkeysgetvalue{/tikz/graphs/placement/chain count}\tikz@temp%
  \c@pgf@counta=\tikz@temp\relax%
  \advance\c@pgf@counta by1\relax%
  \edef\tikz@temp{\the\c@pgf@counta}%
  \pgfkeyslet{/tikz/graphs/placement/chain count}\tikz@temp%
}%


%
% Parse a group
%

\long\def\tikz@lib@graph@parse@group#1{
  \let\tikz@lib@graph@group@qa\pgfutil@empty%
  \let\tikz@lib@graph@group@q\pgfutil@empty%
  \let\tikz@lib@graph@group@c\pgfutil@empty%
  \let\tikz@lib@graph@group@cont\pgfutil@empty%
  \let\tikz@lib@graph@group@conta\pgfutil@empty%
  \tikz@lib@graph@group@check#1\par\pgf@stop@eogroup%
}%



%
% Start of a group
%
\def\tikz@lib@graph@group@check{%
  \pgfutil@ifnextchar[\tikz@lib@graph@group@opt{\tikz@lib@graph@group@opt[]}%]
}%

\def\tikz@lib@graph@group@opt[#1]{%
  \let\tikz@lib@graph@parse@extras\pgfutil@empty%
  \tikzgraphsset{
    @operators=,
    every group/.try,
    @extra group options,
    @extra group options/.style=,%
    #1}%
  \expandafter\tikz@lib@graph@par\tikz@lib@graph@parse@extras%
}%

\tikzgraphsset{
  parse/.code={\expandafter\def\expandafter\tikz@lib@graph@parse@extras\expandafter{\tikz@lib@graph@parse@extras#1}},
}%



%
% Remove \par
%

\def\tikz@lib@graph@par{%
  \pgfutil@ifnextchar\bgroup{\tikz@lib@graph@par@@}{\tikz@lib@graph@par@}%
}%
\long\def\tikz@lib@graph@par@#1\par{%
  \pgfutil@ifnextchar\pgf@stop@eogroup{%
    \expandafter\tikz@lib@graph@quotes\tikz@lib@graph@group@c#1"}{%
    \expandafter\def\expandafter\tikz@lib@graph@group@c\expandafter{\tikz@lib@graph@group@c#1}%
    \tikz@lib@graph@par%
  }%
}%
\long\def\tikz@lib@graph@par@@#1{%
  \expandafter\def\expandafter\tikz@lib@graph@group@c\expandafter{\tikz@lib@graph@group@c{#1}}%
  \tikz@lib@graph@par
}%



%
% Replace ..."..."... by ..."{...}"...
%
\def\tikz@lib@graph@quotes{%
  \pgfutil@ifnextchar\bgroup{\tikz@lib@graph@quotes@@}{\tikz@lib@graph@quotes@}%
}%
\def\tikz@lib@graph@quotes@#1"{%
  \pgfutil@ifnextchar\pgf@stop@eogroup{%
    \tikz@lib@graph@passon{#1}%
  }{%
    \expandafter\def\expandafter\tikz@lib@graph@group@q\expandafter{\tikz@lib@graph@group@q#1"}%
    \tikz@lib@graph@quotes@cont%
  }%
}%

\def\tikz@lib@graph@quotes@cont#1"#2"{%
  \pgfutil@ifnextchar\pgf@stop@eogroup{%
    \tikz@lib@graph@passon{{#1}"#2}}{%
    \expandafter\def\expandafter\tikz@lib@graph@group@q\expandafter{\tikz@lib@graph@group@q{#1}"#2"}%
    \tikz@lib@graph@quotes@cont}%
}%

\def\tikz@lib@graph@quotes@@#1{%
  \expandafter\def\expandafter\tikz@lib@graph@group@q\expandafter{\tikz@lib@graph@group@q{#1}}%
  \tikz@lib@graph@quotes%
}%



%
% Replace ..."..."... by ..."{...}"... (active version)
%
{%
  \catcode`\"=13\relax

  \gdef\tikz@lib@graph@passon#1{\expandafter\tikz@lib@graph@quotesactive\tikz@lib@graph@group@q#1"}%

  \gdef\tikz@lib@graph@quotesactive{%
    \pgfutil@ifnextchar\bgroup{\tikz@lib@graph@quotesactive@@}{\tikz@lib@graph@quotesactive@}%
  }%
  \gdef\tikz@lib@graph@quotesactive@#1"{%
    \pgfutil@ifnextchar\pgf@stop@eogroup{%
      \expandafter\tikz@lib@graph@encloser\tikz@lib@graph@group@qa#1[%
    }{%
      \expandafter\def\expandafter\tikz@lib@graph@group@qa\expandafter{\tikz@lib@graph@group@qa#1"}%
      \tikz@lib@graph@quotesactive@cont%
    }%
  }%

  \gdef\tikz@lib@graph@quotesactive@cont#1"#2"{%
    \pgfutil@ifnextchar\pgf@stop@eogroup{%
      \expandafter\tikz@lib@graph@encloser\tikz@lib@graph@group@qa{#1}"#2[}{%
      \expandafter\def\expandafter\tikz@lib@graph@group@qa\expandafter{\tikz@lib@graph@group@qa{#1}"#2"}%
      \tikz@lib@graph@quotesactive@cont}%
  }%

  \gdef\tikz@lib@graph@quotesactive@@#1{%
    \expandafter\def\expandafter\tikz@lib@graph@group@qa\expandafter{\tikz@lib@graph@group@qa{#1}}%
    \tikz@lib@graph@quotesactive%
  }%
}%




%
% Replace ...[...]... by ...[{...}]...
%
\def\tikz@lib@graph@encloser{%
  \pgfutil@ifnextchar\bgroup{\tikz@lib@graph@encloser@@}{\tikz@lib@graph@encloser@}%
}%
\def\tikz@lib@graph@encloser@#1[{%
  \pgfutil@ifnextchar\pgf@stop@eogroup{%
    \expandafter\tikz@lib@graph@semi\tikz@lib@graph@group@cont#1;%
  }{%
    \expandafter\def\expandafter\tikz@lib@graph@group@cont\expandafter{\tikz@lib@graph@group@cont#1[}%]
    \tikz@lib@graph@encloser@cont%
  }%
}%

\def\tikz@lib@graph@encloser@cont#1]#2[{%
  \pgfutil@ifnextchar\pgf@stop@eogroup{%
    \expandafter\tikz@lib@graph@semi\tikz@lib@graph@group@cont{#1}]#2;}{%
    \expandafter\def\expandafter\tikz@lib@graph@group@cont\expandafter{\tikz@lib@graph@group@cont{#1}]#2[}%
    \tikz@lib@graph@encloser@cont}%
}%

\def\tikz@lib@graph@encloser@@#1{%
  \expandafter\def\expandafter\tikz@lib@graph@group@cont\expandafter{\tikz@lib@graph@group@cont{#1}}%
  \tikz@lib@graph@encloser%
}%




%
% Replace ; by ,
%

\def\tikz@lib@graph@semi{%
  \pgfutil@ifnextchar\bgroup{\tikz@lib@graph@semi@@}{\tikz@lib@graph@semi@}%
}%
\def\tikz@lib@graph@semi@#1;{%
  \pgfutil@ifnextchar\pgf@stop@eogroup{%
    \expandafter\tikz@lib@graph@main@parser\tikz@lib@graph@group@conta#1,}{%
    \expandafter\def\expandafter\tikz@lib@graph@group@conta\expandafter{\tikz@lib@graph@group@conta#1,}%
    \tikz@lib@graph@semi%
  }%
}%
\def\tikz@lib@graph@semi@@#1{%
  \expandafter\def\expandafter\tikz@lib@graph@group@conta\expandafter{\tikz@lib@graph@group@conta{#1}}%
  \tikz@lib@graph@semi%
}%


%
% Main parse
%

\def\tikz@lib@graph@main@parser{%
  \begingroup%
    \pgfkeyssetvalue{/tikz/graphs/placement/local depth}{0}%
    \pgfkeyssetvalue{/tikz/graphs/placement/local width}{0}%
    \let\tikz@lib@graph@stored@actions\pgfutil@empty%
    \let\tikz@lib@graph@node@list\pgfutil@empty% reset
    \tikz@lib@graph@main@parser@start%
}%
\def\tikz@lib@graph@main@parser@start{%
  \pgfutil@ifnextchar\bgroup{\tikz@lib@graph@protect@group}{\tikz@lib@graph@main@parser@cont}%
}%

\def\tikz@lib@graph@protect@group#1{% skip space
  \pgfutil@ifnextchar\relax{\tikz@lib@graph@main@parser@cont{{#1}}}{\tikz@lib@graph@main@parser@cont{{#1}}}%
}%

\def\tikz@lib@graph@main@parser@cont{\tikz@lib@graph@check@quotes\tikz@lib@graph@main@parser@cont@normal}%
\def\tikz@lib@graph@main@parser@cont@normal#1,{%
  \tikz@lib@graph@parse@one#1-\pgf@stop@eodashes%
}%

\def\tikz@lib@graph@parse@one{%
  \pgfutil@ifnextchar\bgroup\tikz@lib@graph@scope\tikz@lib@graph@node%
}%



% Special quoted nodes

\def\tikz@lib@graph@check@quotes#1{%
  \let\tikz@lib@graph@cont@quote#1%
  \pgfutil@ifnextchar"{\begingroup\pgfkeys@temptoks{}\pgfutil@empty\tikz@lib@graph@quote@parser}{\tikz@lib@graph@check@quotes@active}%
}%
{%
  \catcode`\"=13\relax
  \gdef\tikz@lib@graph@check@quotes@active{%
    \pgfutil@ifnextchar"{\begingroup\pgfkeys@temptoks{}\pgfutil@empty\tikz@lib@graph@quote@parser@active}{\tikz@lib@graph@cont@quote}%
  }%
  \gdef\tikz@lib@graph@quote@parser@active"#1"{%
    \pgfkeys@temptoks\expandafter{\the\pgfkeys@temptoks #1}%
    \pgfutil@ifnextchar"{\pgfkeys@temptoks\expandafter{\the\pgfkeys@temptoks "}\tikz@lib@graph@quote@parser@active}{\tikz@lib@graph@quote@parser@done}%
  }%
}%

\def\tikz@lib@graph@quote@parser"#1"{%
  \pgfkeys@temptoks\expandafter{\the\pgfkeys@temptoks #1}%
  \pgfutil@ifnextchar"{\pgfkeys@temptoks\expandafter{\the\pgfkeys@temptoks "}\tikz@lib@graph@quote@parser}{\tikz@lib@graph@quote@parser@done}%
}%

\def\tikz@lib@graph@quote@parser@done{%
  {\expandafter\scantokens\expandafter{%
    \expandafter\expandafter\expandafter\tikzlibgraphactivations\expandafter\expandafter\expandafter\tikzlibgraphdoedef\expandafter{\the\pgfkeys@temptoks}%
  }}%
  {\expandafter\scantokens\expandafter{\expandafter\tikzlibgraphactivationsbrace\expandafter\xdef\expandafter\tikzlibgraphreplaced\expandafter<\tikzlibgraphreplaced>\catcode`\}=2\relax}}%
  \edef\tikzlibgraphreplaced{\expandafter\detokenize\expandafter{\tikzlibgraphreplaced}}
  \pgfutil@ifnextchar/\tikz@lib@graph@quotes@slash\tikz@lib@graph@quotes@no@slash%
}%
\def\tikz@lib@graph@quotes@no@slash{%
  \expandafter\expandafter\expandafter\def\expandafter\expandafter\expandafter%
  \tikz@smuggle\expandafter\expandafter\expandafter{\expandafter\tikzlibgraphreplaced\expandafter/\expandafter{\the\pgfkeys@temptoks}}%
  \expandafter\endgroup\expandafter\tikz@lib@graph@cont@quote\tikz@smuggle%
}%
\def\tikz@lib@graph@quotes@slash/{%
  \pgfutil@ifnextchar/% Ah, double slash...
  {\tikz@lib@graph@quotes@no@slash/}{\expandafter\endgroup\expandafter\tikz@lib@graph@cont@quote\tikzlibgraphreplaced/}%
}%
\def\tikzlibgraphdoedef{\xdef\tikzlibgraphreplaced}%

\def\tikz@lib@prepare@active#1#2#3{%
  {%
    \catcode`#1=13\relax%
    \scantokens{\gdef\tikzlibgraphactivator{\def#2{@#3@}}}%
  }%
  \pgfutil@g@addto@macro\tikzlibgraphactivations{\catcode`#1=13\relax}%
  \expandafter\pgfutil@g@addto@macro\expandafter\tikzlibgraphactivations\expandafter{\tikzlibgraphactivator}%
}%

\let\tikzlibgraphactivations\pgfutil@empty

%
% Remove \outer from \+ for plain TeX
%

\outer\def\tikz@lib@outer@test{\tabalign}%
\ifx\+\tikz@lib@outer@test
  \def\+{\tabalign}
\fi

\tikz@lib@prepare@active{\$}{$}{DOLLAR SIGN}% $
\tikz@lib@prepare@active{\&}{&}{AMPERSAND}%
\tikz@lib@prepare@active{\^}{^}{CIRCUMFLEX ACCENT}%
\tikz@lib@prepare@active{\~}{~}{TILDE}%
\tikz@lib@prepare@active{\_}{_}{LOW LINE}%
\tikz@lib@prepare@active{\|}{|}{VERTICAL LINE}%
\tikz@lib@prepare@active{\[}{[}{LEFT SQUARE BRACKET}%
\tikz@lib@prepare@active{\]}{]}{RIGHT SQUARE BRACKET}%
\tikz@lib@prepare@active{\(}{(}{LEFT PARENTHESIS}%
\tikz@lib@prepare@active{\)}{)}{RIGHT PARENTHESIS}%
\tikz@lib@prepare@active{\/}{/}{SOLIDUS}%
\tikz@lib@prepare@active{\.}{.}{FULL STOP}%
\tikz@lib@prepare@active{\-}{-}{HYPHEN MINUS}%
\tikz@lib@prepare@active{\,}{,}{COMMA}%
\tikz@lib@prepare@active{\+}{+}{PLUS SIGN}%
\tikz@lib@prepare@active{\*}{*}{ASTERISK}%
\tikz@lib@prepare@active{\'}{'}{APOSTROPHE}%
\tikz@lib@prepare@active{\!}{!}{EXCLAMATION MARK}%
\tikz@lib@prepare@active{\"}{"}{QUOTATION MARK}%
\tikz@lib@prepare@active{\:}{:}{COLON}%
\tikz@lib@prepare@active{\;}{;}{SEMINCOLON}%
\tikz@lib@prepare@active{\<}{<}{LESS THAN SIGN}%
\tikz@lib@prepare@active{\=}{=}{EQUALS SIGN}%
\tikz@lib@prepare@active{\>}{>}{GREATER THAN SIGN}%
\tikz@lib@prepare@active{\?}{?}{QUESTION MARK}%
%\tikz@lib@prepare@active{\@}{@}{COMMERCIAL AT}%
\tikz@lib@prepare@active{\`}{`}{GRAVE ACCENT}%
{\catcode`\%=11\catcode`\#=11
\tikz@lib@prepare@active{\%}{%}{PERCENT SIGN}
\tikz@lib@prepare@active{\#}{#}{NUMBER SIGN}
}

\pgfutil@g@addto@macro\tikzlibgraphactivations{\catcode`\\=13\relax}%
{%
  \catcode`\|=0\relax
  \catcode`\\=13\relax
  |pgfutil@g@addto@macro|tikzlibgraphactivations{|def\{@REVERSE SOLIDUS@}}%
}%
{%
  \gdef\tikzlibgraphscommercialat{@COMMERCIAL AT@}%
  \let\g=\pgfutil@g@addto@macro
  \catcode`\@=13\relax
  \g\tikzlibgraphactivations{\catcode`\@=13\relax\let@=\tikzlibgraphscommercialat}%
}%

\def\tikzlibgraphactivationsbrace{%
  \catcode`\{=13\relax%
  \catcode`\}=13\relax%
  \catcode`\<=1\relax%
  \catcode`\>=2\relax%
}%
{%
  \catcode`\{=13\relax%
  \catcode`\}=13\relax%
  \catcode`\<=1\relax%
  \catcode`\>=2\relax%
  \pgfutil@g@addto@macro\tikzlibgraphactivationsbrace<\def{<@LEFT CURLY BRACE@>> %}
  \pgfutil@g@addto@macro\tikzlibgraphactivationsbrace<\def}<@RIGHT CURLY BRACE@>> %{
>%

% A normal node

% First, check for special quote syntax:

\def\tikz@lib@graph@node{\tikz@lib@graph@check@quotes\tikz@lib@graph@node@normal}%

\def\tikz@lib@graph@node@normal{%
  \pgfutil@ifnextchar({\tikz@lib@graph@node@dressed}{\tikz@lib@graph@node@naked}%
}%

\def\tikz@lib@graph@node@dressed(#1){%
  % Add extra braces around nodes to protect dashes
  \tikz@lib@graph@node@naked{(#1)}%
}%

\def\tikz@lib@graph@node@naked#1-{%
  % Detect trailing <
  \tikz@lib@graph@@node#1<\pgf@stop%
}%

\def\tikz@lib@graph@@node#1<#2\pgf@stop%
{
  %
  % #1 will be a node (not a group)
  %
  % Syntax: node name [options]
  %
  % Grab node name
  \tikz@lib@graph@grab@name#1\pgf@stop%
  \tikz@lib@graph@stored@actions%
  \pgfutil@ifnextchar\pgf@stop@eodashes{%
    \tikz@lib@graph@graph@done%
  }{%
    %
    % Now, get arrow kind
    %
    \def\pgf@test{#2}%
    \ifx\pgf@test\pgfutil@empty%
      \expandafter\tikz@lib@graph@no@back@arrow%
    \else%
      \expandafter\tikz@lib@graph@back@arrow%
    \fi%
  }%
}%

\def\tikz@lib@graph@no@back@arrow{%
  \pgfutil@ifnextchar>\tikz@lib@graph@forward@arrow{%
    \pgfutil@ifnextchar-\tikz@lib@graph@undirected@arrow{%
      \pgfutil@ifnextchar!\tikz@lib@graph@no@arrow{%
        \tikzerror{One of the arrow types <-, --, ->, -!- or <-> expected}%
        \tikz@lib@graph@undirected@arrow%
      }%
    }%
  }%
}%

\def\tikz@lib@graph@undirected@arrow-{%
  \def\tikz@lib@graph@arrow@type{--}%
  \tikz@lib@graph@after@arrow%
}%

\def\tikz@lib@graph@forward@arrow>{%
  \def\tikz@lib@graph@arrow@type{->}%
  \tikz@lib@graph@after@arrow%
}%

\def\tikz@lib@graph@bi@arrow>{%
  \def\tikz@lib@graph@arrow@type{<->}%
  \tikz@lib@graph@after@arrow%
}%

\def\tikz@lib@graph@no@arrow!-{%
  \def\tikz@lib@graph@arrow@type{-!-}%
  \tikz@lib@graph@after@arrow%
}%

\def\tikz@lib@graph@back@arrow{%
  \pgfutil@ifnextchar>{\tikz@lib@graph@bi@arrow}{%
    \def\tikz@lib@graph@arrow@type{<-}%
    \tikz@lib@graph@after@arrow%
  }%
}%

\def\tikz@lib@graph@after@arrow{%
  \pgfutil@ifnextchar[{\tikz@lib@graph@after@arrow@opt}{\tikz@lib@graph@after@arrow@opt[]}%]
}%

\def\tikz@lib@graph@after@arrow@opt[#1]{%
  %
  % Ok, first recolor
  %
  \tikzgraphinvokeoperator{recolor source by=source''}
  \tikzgraphinvokeoperator{recolor target by=target'}
  % Save action for next node
  \expandafter\def\expandafter\tikz@lib@graph@stored@actions\expandafter{%
    \expandafter\tikz@lib@graph@joiner\expandafter{\tikz@lib@graph@arrow@type}{#1}}%
  %
  \tikzgdeventgroupcallback{descendants}%
  \tikz@lib@graph@parse@one%
}%

\def\tikz@lib@graph@joiner#1#2{%
  \tikzgraphinvokeoperator{recolor source by=source'}
  \tikzgraphinvokeoperator{recolor source'' by=source}
  {%
    \tikz@enable@edge@quotes%
    \pgfkeyssetvalue{/tikz/graphs/default edge kind}{#1}%
    \pgfkeys{/tikz/graphs/.cd,@operators=,%
      /tikz/graphs/.unknown/.code=\tikz@lib@graph@unknown@edge@option{##1},#2}%
    \pgfkeysgetvalue{/tikz/graphs/@operators}\pgf@temp%
    \ifx\pgf@temp\pgfutil@empty%
      \edef\pgf@temp{\noexpand\pgfkeys{/tikz/graphs/.cd,\pgfkeysvalueof{/tikz/graphs/default edge operator}}}%
      \pgf@temp%
      \pgfkeysgetvalue{/tikz/graphs/@operators}\pgf@temp%
    \fi%
    \pgf@temp%
  }%
  \tikzgraphinvokeoperator{not source',not target'}
}%

\def\tikz@lib@graph@unknown@edge@option#1{%
  \def\tikz@temp{/tikz/graphs/@edges styling/.append=}
  \expandafter\expandafter\expandafter\pgfkeys%
  \expandafter\expandafter\expandafter{\expandafter\tikz@temp\expandafter{\expandafter,\pgfkeyscurrentname={#1}}}
}%

\def\tikz@lib@graph@graph@done\pgf@stop@eodashes{%
    % Get local depth and width outside
    \xdef\tikz@lib@graph@chain@depth{\pgfkeysvalueof{/tikz/graphs/placement/local depth}}
    \xdef\tikz@lib@graph@chain@width{\pgfkeysvalueof{/tikz/graphs/placement/local width}}
    % Get node list outside...
    \expandafter%
  \endgroup%
  \expandafter\expandafter\expandafter\def%
  \expandafter\expandafter\expandafter\tikz@lib@graph@node@list%
  \expandafter\expandafter\expandafter{\expandafter\tikz@lib@graph@node@list\tikz@lib@graph@node@list}%
  % Compute new local depth and width of group...
  \tikz@lib@graph@placement@after@chain@update
  %
  \pgfutil@ifnextchar\pgf@stop@eogroup%
  \tikz@lib@graph@graph@group@done%
  \tikz@lib@graph@main@parser%
}%

\def\tikz@lib@graph@graph@group@done\pgf@stop@eogroup{%
  \pgfkeysvalueof{/tikz/graphs/@operators}%
}%



%
% Handle node
%
\def\tikz@lib@graph@grab@name{%
  \pgfutil@ifnextchar\foreach\tikz@lib@graph@do@foreach\tikz@lib@graph@parse@node@text%
}%

\def\tikz@lib@graph@do@foreach\foreach#1in{%
  \pgfutil@ifnextchar\bgroup{\tikz@lib@graph@do@foreach@normal{#1}}{\def\tikz@temp{#1}\tikz@lib@graph@do@foreach@macro}%
}%
\def\tikz@lib@graph@do@foreach@macro#1{%
  \expandafter\expandafter\expandafter\tikz@lib@graph@do@foreach@normal\expandafter\tikz@temp\expandafter{#1}%
}%

\def\tikz@lib@graph@do@foreach@normal#1#2#3\pgf@stop{%
  % Ok, we do a parse on a foreach loop.
  \begingroup
    \let\tikz@lib@graph@node@list@saved\pgfutil@empty%
    \xdef\tikz@lib@graph@saved@placement{%
      {\pgfkeysvalueof{/tikz/graphs/placement/local depth}}%
      {\pgfkeysvalueof{/tikz/graphs/placement/local width}}%
      {\pgfkeysvalueof{/tikz/graphs/placement/chain count}}%
      {\pgfkeysvalueof{/tikz/graphs/placement/element count}}%
      {\pgfkeysvalueof{/tikz/graphs/placement/width}}%
      {\pgfkeysvalueof{/tikz/graphs/placement/depth}}%
    }%
    \foreach #1 in {#2}%
    {%
      \let\tikz@lib@graph@node@list\tikz@lib@graph@node@list@saved%
      \expandafter\tikz@lib@graph@setup@placement\tikz@lib@graph@saved@placement%
      \tikz@lib@graph@parse@group{#3}%
      \xdef\tikz@lib@graph@saved@placement{%
        {\pgfkeysvalueof{/tikz/graphs/placement/local depth}}%
        {\pgfkeysvalueof{/tikz/graphs/placement/local width}}%
        {\pgfkeysvalueof{/tikz/graphs/placement/chain count}}%
        {\pgfkeysvalueof{/tikz/graphs/placement/element count}}%
        {\pgfkeysvalueof{/tikz/graphs/placement/width}}%
        {\pgfkeysvalueof{/tikz/graphs/placement/depth}}%
      }%
      % TODO: Need to also save hints!
      \global\let\tikz@lib@graph@node@list@saved\tikz@lib@graph@node@list%
    }%
    \expandafter%
  \endgroup%
  \expandafter\expandafter\expandafter\def%
  \expandafter\expandafter\expandafter\tikz@lib@graph@node@list%
  \expandafter\expandafter\expandafter{\expandafter\tikz@lib@graph@node@list\tikz@lib@graph@node@list@saved}%
  \expandafter\tikz@lib@graph@setup@placement\tikz@lib@graph@saved@placement%
}%

\def\tikz@lib@graph@setup@placement#1#2#3#4#5#6{%
  \pgfkeyssetvalue{/tikz/graphs/placement/local depth}{#1}%
  \pgfkeyssetvalue{/tikz/graphs/placement/local width}{#2}%
  \pgfkeyssetvalue{/tikz/graphs/placement/chain count}{#3}%
  \pgfkeyssetvalue{/tikz/graphs/placement/element count}{#4}%
  \pgfkeyssetvalue{/tikz/graphs/placement/width}{#5}%
  \pgfkeyssetvalue{/tikz/graphs/placement/depth}{#6}%
}%

\def\tikz@lib@graph@parse@node@text#1\pgf@stop{%
  %
  % Ok, first test whether #1 contains "//"
  %
  \pgfutil@in@{//}{#1 }
  \ifpgfutil@in@%
    % Ok, a layout node:
    \tikz@lib@parse@layout@node#1\pgf@stop%
  \else%
    \tikz@lib@graph@fake@nodefalse
    \let\tikz@lib@graph@node@parsed\tikz@lib@graph@node@opt@normal%
    \def\tikz@lib@graph@empty@node@parsed{\tikzgdeventcallback{node}{}}%
    \tikz@lib@parse@normal@node#1[\pgf@stop%
  \fi%
}%

\newif\iftikzgraphsautonumbernodes
\newcount\tikz@lib@auto@number

\def\tikz@lib@do@autonumber{%
  \ifx\tikz@lib@graph@name@only\pgfutil@empty%
  \else%
    \edef\tikz@lib@graph@name@only{\tikz@lib@graph@name@only\tikz@lib@auto@sep\the\tikz@lib@auto@number}%
    \global\advance\tikz@lib@auto@number by1\relax%
  \fi%
}%

%
% Parse the normal part of a node (name and, possibly, text after slash)
%

\def\tikz@lib@parse@normal@node#1[{
  %
  % Test whether #1 contains "/" or "__"
  %
  \pgfutil@in@{/}{#1}%
  \ifpgfutil@in@%
    \tikz@lib@parse@node@with@slash#1\pgf@stop%
  \else
    \pgfutil@in@{__}{#1}%
    \ifpgfutil@in@%
      \tikz@lib@parse@node@with@doubleunder#1\pgf@stop%
    \else
      \edef\pgf@marshal{\noexpand\pgfkeys@spdef\noexpand\tikz@lib@graph@name@only{#1}}%
      \pgf@marshal%
      \let\tikz@lib@graph@node@text\tikz@lib@graph@name@only%
      \edef\tikz@lib@graph@name{\tikz@lib@graph@path\tikz@lib@graph@name@only}%
    \fi%
  \fi%
  \tikz@lib@graph@handle@node@cont%
}%

\def\tikz@lib@parse@node@with@slash#1/{
  \pgfkeys@spdef\tikz@lib@graph@name@only{#1}%
  \ifx\tikz@lib@graph@name@only\pgfutil@empty%
    \global\advance\tikz@fig@count by1\relax
    \edef\tikz@lib@graph@name@only{tikz@f@\the\tikz@fig@count}%
  \fi%
  \edef\tikz@lib@graph@name{\tikz@lib@graph@path\tikz@lib@graph@name@only}%
  \pgfutil@ifnextchar"\tikz@lg@slash@quote{\pgfutil@ifnextchar\tikz@active@quotes@token\tikz@lg@slash@quote@active\tikz@lg@slash@cont}
}%
\def\tikz@lg@slash@quote"#1"#2\pgf@stop{%
  \iftikz@handle@active@nodes%
    \def\tikz@lib@graph@node@text{\scantokens{#1}}%
  \else
    \def\tikz@lib@graph@node@text{#1}%
  \fi%
}%
{\catcode`\"=13\relax
  \gdef\tikz@lg@slash@quote@active"#1"#2\pgf@stop{%
  \iftikz@handle@active@nodes%
    \def\tikz@lib@graph@node@text{\scantokens{#1}}%
  \else
    \def\tikz@lib@graph@node@text{#1}%
  \fi%
  }%
}%
\def\tikz@lg@slash@cont#1\pgf@stop{%
  \iftikz@handle@active@nodes%
    \def\tikz@lib@graph@node@text{\scantokens{#1}}%
  \else
    \def\tikz@lib@graph@node@text{#1}%
  \fi%
}%

\def\tikz@lib@parse@node@with@doubleunder#1__{\tikz@lib@parse@node@with@slash{#1}/}%

\def\tikz@lg@find@fresh@name{%
  \edef\tikz@lib@graph@name@only{\tikz@lib@graph@name@only'}%
  \edef\tikz@lib@graph@name{\tikz@lib@graph@path\tikz@lib@graph@name@only}%
  \tikz@lg@if@local@node{\tikz@lib@graph@name}{\tikz@lg@find@fresh@name}{}%
}%


%
% We have now parsed everything up to the opening "[". We continue
%

\def\tikz@lib@graph@handle@node@cont{%
  \iftikzgraphsautonumbernodes%
    \tikz@lib@do@autonumber%
  \fi%
  \iftikz@lib@graph@fresh@node%
    \tikz@lg@if@local@node{\tikz@lib@graph@name}{\tikz@lg@find@fresh@name}{}%
  \fi%
  \let\tikzgraphnodeas\tikzgraphnodeas@default%
  \pgfutil@ifnextchar\pgf@stop{%
    \ifx\tikz@lib@graph@name@only\pgfutil@empty%
      \expandafter\tikz@lib@graph@node@empty@done%
    \else
      \expandafter\tikz@lib@graph@node@opt\expandafter[\expandafter]\expandafter[%
    \fi%
  }{\tikz@lib@graph@node@opt[}%
}%

\def\tikzgraphnodeas@default{%
  \tikz@lib@graph@typesetter%
}%

\let\tikz@lib@graph@empty@node@parsed\relax%

\def\tikz@lib@graph@node@opt[#1]#2[\pgf@stop{%
  \tikz@lib@graph@node@parsed{#1}%
}%

\def\tikz@lib@graph@node@empty@done\pgf@stop{\tikz@lib@graph@empty@node@parsed}%

%
% Parse a layout node
%

\def\tikz@lib@parse@layout@node#1//{%
  \pgf@lib@graph@empty@layout@nodefalse%
  \let\tikz@lib@graph@node@parsed\tikz@lib@layout@node@parsed%
  \let\tikz@lib@graph@empty@node@parsed\tikz@lib@graph@empty@layout@node@parsed
  \tikz@lib@parse@normal@node#1[\pgf@stop%
}%

\newif\ifpgf@lib@graph@empty@layout@node

\def\tikz@lib@layout@node@parsed{%
  \tikz@lib@layout@parse@rest%
}%

\def\tikz@lib@graph@empty@layout@node@parsed{%
  \pgf@lib@graph@empty@layout@nodetrue
  \tikz@lib@layout@parse@rest{}%
}%

\def\tikz@lib@layout@parse@rest#1{%
  \def\tikz@lib@layout@node@options{#1}%
  \pgfutil@ifnextchar[{\tikz@lib@layout@node@opt}{\tikz@lib@layout@node@opt[]}%}
}%

\def\tikz@lib@layout@node@parsed#1{%
  \def\tikz@lib@layout@node@options{#1}%
  \pgfutil@ifnextchar[{\tikz@lib@layout@node@opt}{\tikz@lib@layout@node@opt[]}%}
}%

\def\tikz@lib@layout@node@opt[#1]{%
  \def\tikz@lib@layout@options{#1}%
  \pgfutil@ifnextchar\bgroup{\tikz@lib@layout@start}{\tikzerror{Opening brace at beginning of sublayout expected}}%
}%


%
% Typeset a normal node
%

\def\tikz@lib@graph@node@opt@normal#1{%
  \expandafter\tikz@lib@graph@test@use@list\tikz@lib@graph@name@only\pgf@stop
  \iftikz@lib@graph@use@list%
    % Ok, make a list of the nodes stored in #1:
    \let\tikz@lg@temp\pgfutil@empty%
    \foreach \tikz@lg@node@name in \tikz@lib@graph@use@list {\expandafter\tikz@lib@graph@handle@use\expandafter{\tikz@lg@node@name}}
    % Ok, now add the nodes to the node list
    \expandafter\expandafter\expandafter\def%
    \expandafter\expandafter\expandafter\tikz@lib@graph@node@list%
    \expandafter\expandafter\expandafter{%
      \expandafter\tikz@lib@graph@node@list\tikz@lg@temp}%
    % Then color and initialize them:
    \let\tikz@lg@do\tikz@lib@graph@do@use%
    \tikz@lg@temp%
  \else%
    \edef\tikz@lib@graph@name{\tikz@lib@graph@path\tikz@lib@graph@name@only}%
    \expandafter\ifx\csname tikz@lib@graph@def@\tikz@lib@graph@name@only\endcsname\relax%
      \pgfkeysgetvalue{/tikz/graphs/placement/level}\tikz@temp%
      \c@pgf@counta=\tikz@temp\relax%
      \advance\c@pgf@counta by1\relax%
      \edef\tikz@temp{\the\c@pgf@counta}%
      \pgfkeyslet{/tikz/graphs/placement/level}\tikz@temp%
      \tikzgraphsset{
        level/.try=\pgfkeysvalueof{/tikz/graphs/placement/level},
        level \pgfkeysvalueof{/tikz/graphs/placement/level}/.try
      }
      {%
        \edef\tikz@lib@graph@node@list{\noexpand\tikz@lg@do{\tikz@lib@graph@name}}%
        \global\tikz@lib@graph@node@createdfalse%
        \pgfkeyslet{/tikz/graphs/@operators}\pgfutil@empty%
        \tikz@lib@activate@source@target@edge@syntax%
        \tikz@lg@if@local@node{\tikz@lib@graph@name}%
        {\tikz@lg@local@node@handle{#1}}%
        {%
          \tikz@lg@init@color{\tikz@lib@graph@name}{\tikz@lgc@all@true\tikz@lgc@source@true\tikz@lgc@target@true}%
          \iftikz@lib@graph@all%
            \tikzgraphsset{#1}\pgfkeysvalueof{/tikz/graphs/@operators}%
          \else%
            \pgfkeys{/tikz/graphs/placement/place}%
            \let\tikzgraphnodetext\tikz@lib@graph@node@text%
            \let\tikzgraphnodename\tikz@lib@graph@name@only%
            \let\tikzgraphnodepath\tikz@lib@graph@path%
            \let\tikzgraphnodefullname\tikz@lib@graph@name%
            \iftikz@lib@graph@fake@node%
              {% run options in protected mode...
                \pgfkeys{
                  /tikz/graphs/.cd,%
                  .unknown/.code=,
                  /tikz/graphs/@nodes styling,%
                  #1}%
                \pgfkeysgetvalue{/tikz/graphs/@operators}\tikz@lib@graph@op@save%
                \global\let\tikz@lib@graph@op@save\tikz@lib@graph@op@save%
              }%
            \else%
              \node [%
                name=\tikz@lib@graph@name,%
                execute at end node={%
                  \pgfkeysgetvalue{/tikz/graphs/@operators}\tikz@lib@graph@op@save%
                  \global\let\tikz@lib@graph@op@save\tikz@lib@graph@op@save%
                },%
                graphs/redirect unknown to tikz,
                /tikz/graphs/.cd,%
                /tikz/graphs/@nodes styling,%
                #1]%
                {%
                  \tikzgraphnodeas%
                };%
            \fi%
            \tikz@lib@graph@op@save\global\let\tikz@lib@graph@op@save\relax%%
            \global\tikz@lib@graph@node@createdtrue%
          \fi
        }%
      }%
      \iftikz@lib@graph@node@created\tikz@lib@graph@placement@update\fi%
      \expandafter\expandafter\expandafter\def%
      \expandafter\expandafter\expandafter\tikz@lib@graph@node@list%
      \expandafter\expandafter\expandafter{%
        \expandafter\tikz@lib@graph@node@list\expandafter\tikz@lg@do\expandafter{\tikz@lib@graph@name}}%
      \iftikz@lib@graph@trie\tikzgraphsset{name=\tikz@lib@graph@name@only}\fi%
    \else
      % The name of the node is a graph name
      \tikz@lib@graph@handle@graph{#1}%
    \fi
  \fi%
}%

\newif\iftikz@lib@graph@fake@node

\newif\iftikz@lib@graph@use@list
\def\tikz@lib@graph@test@use@list{%
  \pgfutil@ifnextchar({\tikz@lib@graph@use@list@grap}{\tikz@lib@graph@test@use@list@done}%)
}%

\def\tikz@lib@graph@test@use@list@done#1\pgf@stop{\tikz@lib@graph@use@listfalse}%
\def\tikz@lib@graph@use@list@grap(#1)\pgf@stop{\def\tikz@lib@graph@use@list{#1}\tikz@lib@graph@use@listtrue}%


%
% Typeset a layout node
%

\def\tikz@lib@layout@start#1#2\pgf@stop{%
  \tikz@lib@ensure@gd@loaded%
  % Parameters are:
  %
  % Node name is \tikz@lib@graph@name@only
  % Node text is \tikz@lib@graph@node@text
  % Node options are \tikz@lib@layout@node@options
  % Layout options are \tikz@lib@layout@options
  % Layout is #1
  % Rest is #2 (should be empty or just whitespace)
  \tikz@lib@graph@fake@nodefalse%
  \ifpgf@lib@graph@empty@layout@node%
  \else%
    \ifx\tikz@lib@graph@name\pgfutil@empty%
      \tikzerror{Internal error: Node has no name, but it should!}%
    \else
      \tikz@lib@graph@fake@nodetrue%
      \expandafter\tikz@lib@graph@node@opt@normal\expandafter{\tikz@lib@layout@node@options}%
    \fi%
  \fi%
  \edef\tikz@lib@graph@name{\tikz@lib@graph@path\tikz@lib@graph@name@only}%
  %
  % Prepare tikz node options
  %
  \expandafter\expandafter\expandafter\def%
  \expandafter\expandafter\expandafter\tikz@lib@layout@node@options\expandafter\expandafter\expandafter{\expandafter\tikz@lib@layout@startup@node@options\tikz@lib@layout@node@options,}%
  \begingroup
      % Create a subgraph node, if name is given.
    \iftikz@lib@graph@fake@node%
      \expandafter\expandafter\expandafter%
      \expandafter\expandafter\expandafter\expandafter\tikz@lib@make@subgraph@node%
      \expandafter\expandafter\expandafter%
      \expandafter\expandafter\expandafter\expandafter{%
        \expandafter\expandafter\expandafter\tikz@lib@graph@name%
        \expandafter\expandafter\expandafter}%
      \expandafter\expandafter\expandafter{%
        \expandafter\tikz@lib@graph@node@text\expandafter}\expandafter{\tikz@lib@layout@node@options}
    \fi%
  %
  % Here comes the scope:
  %
  \expandafter\expandafter\expandafter\scope\expandafter\expandafter\expandafter[\expandafter\tikz@lib@layout@node@options@prefix\tikz@lib@layout@options]
    \tikzgdeventgroupcallback{array}%
    \let\tikz@lib@graph@node@list\pgfutil@empty%
    \tikz@lib@graph@start@hint@group%
      \tikz@lib@graph@parse@group{#1}%
    \tikz@lib@graph@end@hint@group%
    \def\tikz@lg@old@col{\tikz@lgc@source@true}%
    \let\tikz@lg@new@col\pgfutil@empty%
    \let\tikz@lg@do\tikz@lg@change@color%
    \tikz@lib@graph@node@list%
    \def\tikz@lg@old@col{\tikz@lgc@target@true}%
    \tikz@lib@graph@node@list%
    \expandafter%
  \endscope\expandafter%
  \endgroup%
  \expandafter\expandafter\expandafter\def%
  \expandafter\expandafter\expandafter\tikz@lib@graph@node@list%
  \expandafter\expandafter\expandafter{\expandafter\tikz@lib@graph@node@list\tikz@lib@graph@node@list}%
  \tikz@lib@graph@hint@aftergroup%
}%
\def\tikz@lib@make@subgraph@node#1#2#3{%
  \pgfgdsubgraphnode{#1}{#3}{\pgfgdsubgraphnodecontents{#2}}%
}%
\def\tikz@lib@layout@startup@node@options{%
  /utils/exec=\tikzlibignorecomparisonsINTERNAL,
  /tikz/graphs/.cd,%
  redirect unknown to tikz,%
  anchor=base,%
  /tikz/every subgraph node/.try,%
}%
\def\tikzlibignorecomparisonsINTERNAL{%
  \pgfkeys{/handlers/first char syntax=true}
  \pgfkeyssetvalue{/handlers/first char syntax/the character >}{\pgfutil@gobble}%
  \pgfkeyssetvalue{/handlers/first char syntax/the character <}{\pgfutil@gobble}%
}%
\def\tikz@lib@layout@node@options@prefix{graphs/.cd,}%

\ifx\tikz@lib@ensure@gd@loaded\pgfutil@undefined%
\def\tikz@lib@ensure@gd@loaded{\tikzerror{You must say \string\usetikzlibrary{graphdrawing} to use the (sub)layout syntax}}%
\fi

\def\tikz@lg@local@node@handle#1{%
  %
  % Handle late options and operators
  \tikzgraphsset{source,target,.unknown/.code=,#1}%
  \tikzgdlatenodeoptionacallback{\tikz@lib@graph@name}%
  \node also[graphs/redirect unknown to tikz,/tikz/graphs/.cd,#1](\tikz@lib@graph@name);%
  \pgfkeysvalueof{/tikz/graphs/@operators}%
}%

\tikzgraphsset{redirect unknown to tikz/.style={
    /tikz/graphs/.unknown/.code={%
      \let\tikz@key\pgfkeyscurrentname%
      \pgfkeys{tikz/.cd,\tikz@key={##1},/tikz/graphs/.cd}%
    }}
}%

\def\tikz@lib@activate@source@target@edge@syntax{%
  \pgfkeys{/handlers/first char syntax=true}
  \pgfkeyssetvalue{/handlers/first char syntax/the character >}{\tikz@lg@parse@more}%
  \pgfkeyssetvalue{/handlers/first char syntax/the character <}{\tikz@lg@parse@less}%
}%

\def\tikz@lg@parse@less#1{\tikz@lg@parse@less@#1\pgf@stop}%
\def\tikz@lg@parse@less@<{\pgfutil@ifnextchar"{\tikz@lg@parse@quote{source}}{\tikz@lg@parse@noquote{source}}}%
\def\tikz@lg@parse@more#1{\tikz@lg@parse@more@#1\pgf@stop}%
\def\tikz@lg@parse@more@>{\pgfutil@ifnextchar"{\tikz@lg@parse@quote{target}}{\tikz@lg@parse@noquote{target}}}%

\def\tikz@lg@parse@noquote#1#2\pgf@stop{{\tikzgraphsset{#1 edge style={#2}}}}%
\def\tikz@lg@parse@quote#1#2\pgf@stop{%
  {\tikzgraphsset{/tikz/node quotes mean={#1 edge node={node [every edge quotes,##2]{##1}}},/utils/exec=\tikz@enable@node@quotes,#2}}%
}%

\tikzgraphsset{
  clear >/.style=target edge clear,
  clear </.style=source edge clear
}%


% Positioning

\def\tikz@lib@graph@x{0}%
\def\tikz@lib@graph@y{0}%

\tikzgraphsset{
  x/.code=\def\tikz@lib@graph@x{#1}\tikz@lib@graphs@check@at,
  y/.code=\def\tikz@lib@graph@y{#1}\tikz@lib@graphs@check@at
}%
\def\tikz@lib@graphs@check@at{%
  \expandafter\expandafter\expandafter\def%
  \expandafter\expandafter\expandafter\tikz@temp%
  \expandafter\expandafter\expandafter{%
    \expandafter\expandafter\expandafter(%
    \expandafter\expandafter\expandafter{\expandafter\tikz@lib@graph@x\expandafter}%
    \expandafter,\expandafter{\tikz@lib@graph@y})}%
  \pgfqkeys{/tikz}{at/.expand once=\tikz@temp}%
}%


\newif\iftikz@lib@graph@trie
\tikzgraphsset{
  trie/.is if=tikz@lib@graph@trie,
  put node text on incoming
  edges/.style={nodes={/utils/exec=\tikz@lg@make@edge@node{target}{#1}}},
  put node text on outgoing
  edges/.style={nodes={/utils/exec=\tikz@lg@make@edge@node{source}{#1}}},
  edge quotes/.style={/tikz/every edge quotes/.style={#1}},
  edge quotes center/.style={edge quotes={anchor=center}},
  edge quotes mid/.style={edge quotes={anchor=mid}}
}%

\def\tikz@lg@make@edge@node#1#2{%
  \def\pgf@marshal{node[#2]}%
  \expandafter\expandafter\expandafter\def\expandafter\expandafter\expandafter\pgf@marshal\expandafter\expandafter\expandafter{\expandafter\pgf@marshal\expandafter{\tikzgraphnodetext}}%
  \pgfkeysalso{#1 edge node/.expand once=\pgf@marshal,as=}%
}%

\newif\iftikz@lib@graph@fresh@node
\tikzgraphsset{fresh nodes/.is if=tikz@lib@graph@fresh@node}%

\tikzgraphsset{number nodes/.code=%
  \pgfmathsetcount\tikz@lib@auto@number{#1}%
  \tikzgraphsautonumbernodestrue,%
  number nodes/.default=1,%
  number nodes sep/.code=\def\tikz@lib@auto@sep{#1}
}%
\def\tikz@lib@auto@sep{\space}%

\newif\iftikz@lib@graph@node@created

\def\tikz@lib@graph@handle@use#1{%
  % Is #1 the name of a node set?
  \expandafter\let\expandafter\pgf@temp\csname tikz@lg@node@set #1\endcsname
  \ifx\pgf@temp\relax
    \pgfutil@g@addto@macro\tikz@lg@temp{\tikz@lg@do{#1}}
  \else%
    \expandafter\pgfutil@g@addto@macro\expandafter\tikz@lg@temp\expandafter{\pgf@temp}
  \fi
}%

\def\tikz@lib@graph@do@use#1{%
  \tikz@lg@init@color{#1}{\tikz@lgc@all@true\tikz@lgc@source@true\tikz@lgc@target@true}%
}%

\tikzgraphsset{
  typeset/.store in=\tikz@lib@graph@typesetter,
  math nodes/.style={/tikz/graphs/typeset=$\tikzgraphnodetext$},
  empty nodes/.style={/tikz/graphs/typeset=},
  typeset=\tikzgraphnodetext
}%


%
% Handle scope
%
\def\tikz@lib@graph@scope#1{
  \begingroup%
    \tikzgdeventgroupcallback{array}%
    \let\tikz@lib@graph@node@list\pgfutil@empty%
    \tikz@lib@graph@start@hint@group%
      \tikz@lib@graph@parse@group{#1}%
    \tikz@lib@graph@end@hint@group%
    \expandafter%
  \endgroup%
  \expandafter\expandafter\expandafter\def%
  \expandafter\expandafter\expandafter\tikz@lib@graph@node@list%
  \expandafter\expandafter\expandafter{\expandafter\tikz@lib@graph@node@list\tikz@lib@graph@node@list}%
  \tikz@lib@graph@hint@aftergroup%
  \tikz@lib@graph@stored@actions%
  \pgfutil@ifnextchar-{\tikz@lib@graph@scope@minus}{%
    \pgfutil@ifnextchar<{\tikz@lib@graph@scope@less}{%
      \tikzerror{One of the arrow types <-, --, ->, -!-, or <-> expected}%
    }%
  }%
}%

\def\tikz@lib@graph@scope@minus-{
  \pgfutil@ifnextchar>\tikz@lib@graph@forward@arrow{%
    \pgfutil@ifnextchar-\tikz@lib@graph@undirected@arrow{%
      \pgfutil@ifnextchar!\tikz@lib@graph@no@arrow{%
        \pgfutil@ifnextchar\pgf@stop@eodashes\tikz@lib@graph@graph@done{%
          \tikzerror{One of the arrow types <-, --, ->, -!-, or <-> expected}%
          \tikz@lib@graph@undirected@arrow%
        }%
      }%
    }%
  }%
}%

\def\tikz@lib@graph@scope@less<-{\tikz@lib@graph@back@arrow}%




%
% Predefining graphs
%

\tikzgraphsset{
  declare/.code 2 args={\expandafter\def\csname tikz@lib@graph@def@#1\endcsname{\tikz@lib@graph@do@graph{#2}}}%
}%

\def\tikz@lib@graph@handle@graph#1{%
  \begingroup%
    \let\tikz@lib@graph@node@list\pgfutil@empty%
    \tikzgraphsset{@extra group options/.style={#1}}%
    \tikz@lib@graph@start@hint@group%
      \csname tikz@lib@graph@def@\tikz@lib@graph@name@only\endcsname%
    \tikz@lib@graph@end@hint@group%
    \expandafter%
  \endgroup%
  \expandafter\expandafter\expandafter\def%
  \expandafter\expandafter\expandafter\tikz@lib@graph@node@list%
  \expandafter\expandafter\expandafter{\expandafter\tikz@lib@graph@node@list\tikz@lib@graph@node@list}%
  \tikz@lib@graph@hint@aftergroup%
}%

\def\tikz@lib@graph@do@graph#1{%
  \tikz@lib@graph@parse@group{#1}%
}%

\let\tikz@lib@graph@path\pgfutil@empty

\tikzgraphsset{
  name separator/.store in=\tikz@lib@graph@name@separator,
  name separator=\space,
  name/.code={%
    \edef\tikz@lib@graph@path{\tikz@lib@graph@path#1\tikz@lib@graph@name@separator}%
  }%
}%


%
% Colors
%
\def\tikz@lg@newif{\csname newif\endcsname}%
\tikzgraphsset{
  as/.code=\def\tikzgraphnodeas{#1},%
  color class/.style={%
    /utils/exec=\expandafter\tikz@lg@newif\csname iftikz@lgc@#1@\endcsname,
    #1/.style={operator={%
      \edef\tikz@lg@col{\expandafter\noexpand\csname tikz@lgc@#1@true\endcsname}%
      \let\tikz@lg@do\tikz@lg@colorize%
      \tikz@lib@graph@node@list%
    }},
    not #1/.style={operator={%
      \edef\tikz@lg@old@col{\expandafter\noexpand\csname tikz@lgc@#1@true\endcsname}%
      \let\tikz@lg@new@col\pgfutil@empty%
      \let\tikz@lg@do\tikz@lg@change@color%
      \tikz@lib@graph@node@list%
    }},
    recolor #1 by/.style={operator={%
      \edef\tikz@lg@old@col{\expandafter\noexpand\csname tikz@lgc@#1@true\endcsname}%
      \edef\tikz@lg@new@col{\expandafter\noexpand\csname tikz@lgc@##1@true\endcsname}%
      \let\tikz@lg@do\tikz@lg@change@color%
      \tikz@lib@graph@node@list%
    }},
    !#1/.style=not #1,
  },
  color class=source,
  color class=source',
  color class=source'',
  color class=target,
  color class=target',
  color class=all
}%

\def\tikz@lg@init@color#1#2{%
  \expandafter\gdef\csname tikz@lgc@#1\endcsname{#2}%
}%

\def\tikz@lib@graph@cleanup#1{%
  \expandafter\global\expandafter\let\csname tikz@lgc@#1\endcsname\relax%
  \ifcsname tikz@lgca@@#1\endcsname\expandafter\global\expandafter\let\csname tikz@lgca@@#1\endcsname\relax\fi%
  \ifcsname tikz@lgcb@@#1\endcsname\expandafter\global\expandafter\let\csname tikz@lgcb@@#1\endcsname\relax\fi%
}%

\def\tikz@lg@colorize#1{%
  \expandafter\let\expandafter\pgf@temp\csname tikz@lgc@#1\endcsname%
  \expandafter\expandafter\expandafter\def%
  \expandafter\expandafter\expandafter\pgf@temp%
  \expandafter\expandafter\expandafter{%
    \expandafter\tikz@lg@col\pgf@temp}%
  \expandafter\global\expandafter\let\csname tikz@lgc@#1\endcsname\pgf@temp%
}%

\def\tikz@lg@change@color#1{%
  \def\tikz@lg@temp@save{#1}%
  \let\tikz@lg@collect\pgfutil@empty%
  \expandafter\let\expandafter\pgf@temp\csname tikz@lgc@#1\endcsname%
  \expandafter\tikz@lg@change@check\pgf@temp\pgf@stop%
}%
\def\tikz@lg@change@check#1{%
  \ifx#1\pgf@stop%
    \tikz@lg@change@write@back%
  \else%
    \def\pgf@temp{#1}%
    \ifx\pgf@temp\tikz@lg@old@col%
      \expandafter\tikz@lg@change@add\expandafter{\tikz@lg@new@col}% Found!
    \else%
      \tikz@lg@change@add{#1}%
    \fi%
    \expandafter\tikz@lg@change@check
  \fi%
}%
\def\tikz@lg@change@add#1{%
  \expandafter\def\expandafter\tikz@lg@collect\expandafter{\tikz@lg@collect#1}%
}%

\def\tikz@lg@change@write@back{%
  \expandafter\global\expandafter\let\csname tikz@lgc@\tikz@lg@temp@save\endcsname\tikz@lg@collect%
}%



\def\tikz@lg@if@has@color#1#2#3#4{%
  {%
    \csname tikz@lgc@#1\endcsname%
    \expandafter\let\expandafter\pgf@temp\csname iftikz@lgc@#2@\endcsname%
    \ifx\pgf@temp\relax%
      \tikz@lib@reset@temp%
    \fi%
    \pgf@temp%
      \global\tikz@color@testtrue%
    \else%
      \global\tikz@color@testfalse%
    \fi%
  }%
  \iftikz@color@test#3\else#4\fi%
}%
\newif\iftikz@color@test

\def\tikz@lg@if@local@node#1#2#3{\expandafter\ifx\csname tikz@lgc@#1\endcsname\relax#3\else#2\fi}%

\def\tikz@lib@reset@temp{\let\pgf@temp\iffalse}%

%
% Handle connection annotations
%
\def\tikz@lib@annotate@#1#2#3#4{%
  \expandafter\ifx\csname tikz@lgc#1@@#2\endcsname\relax%
    \expandafter\gdef\csname tikz@lgc#1@@#2\endcsname{{#3}{#4}}%
  \else%
    \expandafter\expandafter\expandafter\tikz@lib@annotate@read\csname tikz@lgc#1@@#2\endcsname{#3}{#4}%
    \expandafter\global\expandafter\let\csname tikz@lgc#1@@#2\endcsname\pgf@temp%
  \fi%
}%
\def\tikz@lib@annotate@read#1#2#3#4{\def\pgf@temp{{#1,#3}{#2#4}}}%

\tikzgraphsset{
  source edge style/.code=\tikz@lib@annotate@{a}{\tikz@lib@graph@name}{#1}{},
  source edge node/.code=\tikz@lib@annotate@{a}{\tikz@lib@graph@name}{}{#1},
  source edge clear/.code={
    \expandafter\global\expandafter\let\csname tikz@lgca@@\tikz@lib@graph@name\endcsname\relax%
  },
  target edge style/.code=\tikz@lib@annotate@{b}{\tikz@lib@graph@name}{#1}{},
  target edge node/.code=\tikz@lib@annotate@{b}{\tikz@lib@graph@name}{}{#1},
  target edge clear/.code={%
    \expandafter\global\expandafter\let\csname tikz@lgcb@@\tikz@lib@graph@name\endcsname\relax%
  }
}%



% Packing a node list: Replace the node list by a new node list where
% each node is mentioned at most once.

\def\tikz@lib@graph@pack@node@list{%
  {%
    \let\tikz@lg@packed\pgfutil@empty%
    \let\tikz@lg@do\tikz@lg@packer%
    \tikz@lib@graph@node@list%
    \expandafter
  }\expandafter%
  \def\expandafter\tikz@lib@graph@node@list\expandafter{\tikz@lg@packed}%
}%

\def\tikz@lg@packer#1{%
  \expandafter\ifx\csname tikz@lg@p@#1\endcsname\pgf@stop%
  \else%
    \expandafter\let\csname tikz@lg@p@#1\endcsname\pgf@stop%
    \expandafter\def\expandafter\tikz@lg@packed\expandafter{\tikz@lg@packed\tikz@lg@do{#1}}
  \fi
}%

%
% Color functions
%

% Do something for all nodes having a certain color
%
% #1 = the color name
% #2 = a macro
%
% Description:
%
% For each node having color #1, the macro #2 will be called. This
% macro should take a single parameter, which will be set
% to the node's name.

\def\tikzgraphforeachcolorednode#1#2{%
  \tikz@lib@graph@pack@node@list%
  \expandafter\def\expandafter\iftikz@lib@graph@color@picker\expandafter{\csname iftikz@lgc@#1@\endcsname}%
  \let\tikz@lib@graph@action#2%
  \let\tikz@lg@do\tikz@lg@pick%
  \tikz@lib@graph@node@list%
}%
\def\tikz@lg@pick#1{
  {%
    \csname tikz@lgc@#1\endcsname%
    \iftikz@lib@graph@color@picker
      \global\tikz@color@testtrue%
    \else%
      \global\tikz@color@testfalse%
    \fi%
  }%
  \iftikz@color@test\tikz@lib@graph@action{#1}\fi%
}%



% Prepare a color
%
% #1 is the color name
% #2 is a counter
% #3 is a prefix
%
% Description:
%
% You can call this function inside a connector. It will do the
% following: First, its counts how many nodes exist that have color
% #1. This number is stored in the counter passed as #2. Furthermore,
% let <name> be the name of the <i>-th vertex that has color #1. Then, a
% macro called \#3<i> will store <name>. For instance, if #1 is "red"
% and the third red node is called foo and if #3 is "bar", then a
% macro called "\bar3" is set to "foo" as if you had said
% "\expandafter\def\csname bar3\endcsname{foo}".
%
% The bottom line of all this is that after a preparation you can
% easily iterate over nodes having a certain color. If you wish to
% iterate over a single color, it will be quicker and easier to call
% \tikzgraphforeachcolorednode, but if you need to iterate over two
% colors simultaneously, it will be better to first prepare the color.

\def\tikzgraphpreparecolor#1#2#3{%
  \let\tikz@lib@graph@count#2%
  \tikz@lib@graph@count0\relax
  \def\tikz@lib@graph@prefix{#3}%
  \tikzgraphforeachcolorednode{#1}\tikz@lib@graph@prepare%
}%

\def\tikz@lib@graph@prepare#1{%
  \advance\tikz@lib@graph@count by1\relax%
  \expandafter\def\csname\tikz@lib@graph@prefix\the\tikz@lib@graph@count\endcsname{#1}%
}%




%
% The bipartite connector
%

\tikzgraphsset{
  complete bipartite/.style 2 args={operator={
    \def\tikz@lg@shoreb{#2}%
    \tikzgraphforeachcolorednode{#1}\tikz@lib@graph@bipartite@outer
  }},
  complete bipartite/.default={target'}{source'},
  induced complete bipartite/.style 2 args={
    induced independent set={#1},
    induced independent set={#2},
    complete bipartite={#1}{#2}
  },
  induced complete bipartite/.default={target'}{source'},
}%

\def\tikz@lib@graph@bipartite@outer#1{%
  \def\tikz@lib@graph@from{#1}%
  {%
    \tikzgraphforeachcolorednode{\tikz@lg@shoreb}\tikz@lib@graph@bipartite@inner%
  }%
}%

\def\tikz@lib@graph@bipartite@inner#1{%
  \def\pgf@temp{#1}%
  \ifx\pgf@temp\tikz@lib@graph@from\else%
    \tikz@lib@graph@default@new@edge{\tikz@lib@graph@from}{#1}%
  \fi%
}%

\def\tikz@lib@graph@default@new@edge#1#2{%
  \pgfkeysgetvalue{/tikz/graphs/@edges styling}\pgf@temp
  \pgfkeysgetvalue{/tikz/graphs/@edges node}\pgf@temp@b
  \let\tikz@lib@add@temp\pgfutil@empty%
  \tikz@lib@graph@add@edge@annotations{a}{#1}
  \tikz@lib@graph@add@edge@annotations{b}{#2}
  \expandafter\expandafter\expandafter\def%
  \expandafter\expandafter\expandafter\pgf@temp%
  \expandafter\expandafter\expandafter{\expandafter\pgf@temp\tikz@lib@add@temp}%
  \expandafter\expandafter\expandafter\tikz@lib@graph@default@new@edge@%
  \expandafter\expandafter\expandafter{\expandafter\pgf@temp\expandafter}\expandafter{\pgf@temp@b}{#1}{#2}%
}%
\def\tikz@lib@graph@default@new@edge@#1#2#3#4{%
  \iftikz@lib@graph@simple%
    \edef\tikz@temp{{\pgfkeysvalueof{/tikz/graphs/default edge kind}}{#3}{#4}}
    \expandafter\tikz@lib@graph@set@simple@edge\tikz@temp{#1}{#2}%
  \else%
    \pgfkeys{/tikz/graphs/.cd,new \pgfkeysvalueof{/tikz/graphs/default edge kind}={#3}{#4}{#1}{#2}}%
  \fi%
}%
\def\tikz@lib@graph@add@edge@annotations#1#2{%
  \ifcsname tikz@lgc#1@@#2\endcsname%
    \expandafter\let\expandafter\tikz@lg@temp@\csname tikz@lgc#1@@#2\endcsname%
    \ifx\tikz@lg@temp@\relax\else
      \expandafter\tikz@lib@graph@add@edge@annotations@\tikz@lg@temp@%
      \let\tikz@lib@add@temp\tikz@lib@final@edge@style
    \fi%
  \fi%
}%
\def\tikz@lib@graph@add@edge@annotations@#1#2{
  \expandafter\def\expandafter\pgf@temp\expandafter{\pgf@temp,#1}%
  \expandafter\def\expandafter\pgf@temp@b\expandafter{\pgf@temp@b#2}%
}%
\def\tikz@lib@final@edge@style{,after source and target edge/.try}%

%
% The clique connector
%

\tikzgraphsset{
  clique/.style={operator={
      \tikzgraphpreparecolor{#1}\c@pgf@counta{tikz@lg}%
      \tikz@lg@clique@loop%
    }},
  clique/.default=all
}%

\def\tikz@lg@clique@loop{%
  \ifnum\c@pgf@counta=0\relax%
  \else
    \c@pgf@countb=\c@pgf@counta\relax%
    \tikz@lg@clique@loop@inner%
    \advance\c@pgf@counta by-1\relax%
    \expandafter\tikz@lg@clique@loop%
  \fi%
}%

\def\tikz@lg@clique@loop@inner{%
  \advance\c@pgf@countb by-1\relax%
  \ifnum\c@pgf@countb>0\relax%
    \tikz@lib@graph@default@new@edge{\csname tikz@lg\the\c@pgf@countb\endcsname}{\csname tikz@lg\the\c@pgf@counta\endcsname}%
    \expandafter\tikz@lg@clique@loop@inner%
  \fi%
}%


%
% The independent set connector
%

\tikzgraphsset{
  induced independent set/.style={operator={
      \pgfkeysgetvalue{/tikz/graphs/default edge kind}\tikz@lg@default%
      \pgfkeyssetvalue{/tikz/graphs/default edge kind}{-!-}%
      \tikzgraphpreparecolor{#1}\c@pgf@counta{tikz@lg}%
      \tikz@lg@indep@loop%
      \pgfkeyslet{/tikz/graphs/default edge kind}\tikz@lg@default%
    }},
  induced independent set/.default=all
}%

\def\tikz@lg@indep@loop{%
  \ifnum\c@pgf@counta=0\relax%
  \else
    \c@pgf@countb=\c@pgf@counta\relax%
    \tikz@lg@indep@loop@inner%
    \advance\c@pgf@counta by-1\relax%
    \expandafter\tikz@lg@indep@loop%
  \fi%
}%

\def\tikz@lg@indep@loop@inner{%
  \advance\c@pgf@countb by-1\relax%
  \ifnum\c@pgf@countb>0\relax%
    \tikz@lib@graph@default@new@edge{\csname tikz@lg\the\c@pgf@counta\endcsname}{\csname tikz@lg\the\c@pgf@countb\endcsname}%
    \expandafter\tikz@lg@indep@loop@inner%
  \fi%
}%


%
% The path connector
%

\tikzgraphsset{
  path/.style={operator={%
      \let\tikz@lg@prev\relax%
        \tikzgraphforeachcolorednode{#1}\tikz@lib@graph@path@do%
  }},
  path/.default=all,
  induced path/.style={induced independent set={#1},path={#1}},
  induced path/.default=all,
}%

\def\tikz@lib@graph@path@do#1{%
  \ifx\tikz@lg@prev\relax%
  \else%
    \tikz@lib@graph@default@new@edge{\tikz@lg@prev}{#1}%
  \fi
  \def\tikz@lg@prev{#1}%
}%


%
% The cycle connector
%

\tikzgraphsset{
  cycle/.style={operator={%
    \let\tikz@lg@prev\relax%
    \let\tikz@lg@first\relax%
    \tikzgraphforeachcolorednode{#1}\tikz@lib@graph@cycle@do%
    \ifx\tikz@lg@first\relax%
    \else%
      \tikz@lib@graph@default@new@edge{\tikz@lg@prev}{\tikz@lg@first}%
    \fi%
  }},
  cycle/.default=all,
  induced cycle/.style={induced independent set={#1},cycle={#1}},
  induced cycle/.default=all,
}%

\def\tikz@lib@graph@cycle@do#1{%
  \ifx\tikz@lg@prev\relax%
    \def\tikz@lg@prev{#1}%
    \let\tikz@lg@first\tikz@lg@prev%
  \else%
    \tikz@lib@graph@default@new@edge{\tikz@lg@prev}{#1}%
    \def\tikz@lg@prev{#1}%
  \fi%
}%




%
% The matching and star connector
%

\tikzgraphsset{
  matching and star/.style 2 args={operator=%
    {%
      \tikzgraphpreparecolor{#1}\c@pgf@counta{tikz@lg}
      \c@pgf@countb=0\relax%
      \let\tikz@lg@prev\relax
      \tikzgraphforeachcolorednode{#2}\tikz@lib@graph@flow@do%
      \tikz@lib@graph@flow@rest%
    }%
  },
  matching and star/.default={target'}{source'}
}%

\def\tikz@lib@graph@flow@do#1{%
  \advance\c@pgf@countb by1\relax%
  \ifnum\c@pgf@countb>\c@pgf@counta\relax%
    \c@pgf@countb=\c@pgf@counta\relax%
  \fi%
  \ifnum\c@pgf@countb>0\relax%
    \tikz@lib@graph@default@new@edge{\csname tikz@lg\the\c@pgf@countb\endcsname}{#1}%
  \fi%
  \def\tikz@lg@prev{#1}%
}%

\def\tikz@lib@graph@flow@rest{%
  \ifnum\c@pgf@countb<\c@pgf@counta\relax%
    \ifnum\c@pgf@countb>0\relax%
      \advance\c@pgf@countb by1\relax%
      \tikz@lib@graph@default@new@edge{\csname tikz@lg\the\c@pgf@countb\endcsname}{\tikz@lg@prev}%
      \expandafter\tikz@lib@graph@flow@rest%
    \fi%
  \fi%
}%



%
% The matching connector
%

\tikzgraphsset{
  matching/.style 2 args={operator=
    {%
      \tikzgraphpreparecolor{#1}\c@pgf@counta{tikz@lg}
      \c@pgf@countb=0\relax%
      \tikzgraphforeachcolorednode{#2}\tikz@lib@graph@matching@do%
    }%
  },
  matching/.default={target'}{source'}
}%

\def\tikz@lib@graph@matching@do#1{%
  \advance\c@pgf@countb by1\relax%
  \ifnum\c@pgf@countb>\c@pgf@counta\relax%
  \else%
    \tikz@lib@graph@default@new@edge{\csname tikz@lg\the\c@pgf@countb\endcsname}{#1}%
  \fi%
}%





%
% The butterfly connector
%

\tikzgraphsset{
  butterfly/.style={operator=
    {}{%
      \pgfkeys{/tikz/graphs/butterfly/.cd,#1}%
      \ifnum\pgfkeysvalueof{/tikz/graphs/butterfly/level}=0\relax%
        \tikzgraphinvokeoperator{matching and star={\pgfkeysvalueof{/tikz/graphs/butterfly/from}}{\pgfkeysvalueof{/tikz/graphs/butterfly/to}}}%
      \else%
        {%
          \tikzgraphpreparecolor{\pgfkeysvalueof{/tikz/graphs/butterfly/from}}\c@pgf@counta{tikz@lg}
          \c@pgf@countb=0\relax%
          \tikzgraphforeachcolorednode{\pgfkeysvalueof{/tikz/graphs/butterfly/to}}\tikz@lib@graph@butterfly@do%
        }%
        \iftikz@butterfly@prime\else\tikzgraphinvokeoperator{matching and star={\pgfkeysvalueof{/tikz/graphs/butterfly/from}}{\pgfkeysvalueof{/tikz/graphs/butterfly/to}}}\fi%
      \fi%
    }%
  },
  butterfly/.default=,
  butterfly/level/.initial=1,
  butterfly/from/.initial=target',
  butterfly/to/.initial=source',
  butterfly'/.style={operator={}{\tikz@butterfly@primetrue\pgfkeysalso{butterfly={#1}}}},
  butterfly'/.default=,
}%

\newif\iftikz@butterfly@prime

\def\tikz@lib@graph@butterfly@do#1{%
  \advance\c@pgf@countb by1\relax%
  % Compute other side...
  \c@pgf@countc=\pgfkeysvalueof{/tikz/graphs/butterfly/level}\relax%
  {%
    % Computer countb mod (2level)
    \count0=\c@pgf@countc\relax%
    \multiply\count0 by2\relax%
    \count1=\c@pgf@countb\relax%
    \advance\count1 by-1\relax%
    \count2=\count1\relax%
    \count3=\count1\relax%
    \divide\count1 by\count0\relax%
    \multiply\count1 by\count0\relax%
    \advance\count2 by-\count1\relax%
    % count0 = 2*level
    % count2 = countb mod (2level) (starting with 0)
    % count1 = countb - count2 (starting with 0)
    \ifnum\count2<\c@pgf@countc\relax%
      \advance\count3 by \c@pgf@countc\relax%
    \else%
      \advance\count3 by -\c@pgf@countc\relax%
    \fi%
    \expandafter%
  }%
  \expandafter\c@pgf@countc\the\count3\relax%
  \advance\c@pgf@countc by1\relax%
  \ifnum\c@pgf@countc>\c@pgf@counta\relax%
    \c@pgf@countc=\c@pgf@counta\relax%
  \fi%
  \tikz@lib@graph@default@new@edge{\csname tikz@lg\the\c@pgf@countc\endcsname}{#1}%
}%




%
% The no edges connector
%
\tikzgraphsset{no edges/.style={operator=\relax}}%




%
% The grid connector
%
\tikzgraphsset{
  grid/.style={operator={
      \tikzgraphpreparecolor{#1}\c@pgf@counta{tikz@lg}%
      \c@pgf@countb=0\relax%
      \tikzgraphpreparewrapafter%
      \tikz@lg@grid@loop%
    }%
  },
  grid/.default=all
}%

\def\tikzgraphpreparewrapafter{%
  \pgfkeysgetvalue{/tikz/graphs/wrap after}\tikz@temp%
  \ifnum\tikz@temp=0\relax%
    \pgfmathparse{sqrt(\tikzgraphVnum)}%
    \pgfmathint{\pgfmathresult}%
    \edef\tikzgraphwrapafter{\pgfmathresult}%
  \else%
    \edef\tikzgraphwrapafter{\tikz@temp}
  \fi%
}%

\def\tikz@lg@grid@loop{%
  \ifnum\c@pgf@counta=0\relax%
  \else%
    \advance\c@pgf@countb by 1\relax%
    \pgfmathmod{\c@pgf@countb-1}{\tikzgraphwrapafter}%
    \pgfmathint{\pgfmathresult}%
    \ifnum\pgfmathresult>0%
      \pgfmathsubtract{\c@pgf@countb}{1}%
      \pgfmathint{\pgfmathresult}%
      \tikz@lib@graph@default@new@edge{\csname tikz@lg\the\c@pgf@countb\endcsname}{\csname tikz@lg\pgfmathresult\endcsname}%
    \fi%
    \pgfmathparse{\c@pgf@countb <= \tikzgraphwrapafter}%
    \ifnum\pgfmathresult=0%
      \pgfmathsubtract{\c@pgf@countb}{\tikzgraphwrapafter}%
      \pgfmathint{\pgfmathresult}%
      \tikz@lib@graph@default@new@edge{\csname tikz@lg\pgfmathresult\endcsname}{\csname tikz@lg\the\c@pgf@countb\endcsname}%
    \fi%
    \advance\c@pgf@counta by-1\relax%
    \expandafter\tikz@lg@grid@loop%
  \fi%
}%




% Positioning
%
% It is not the job of the graph library to compute good positions for
% nodes in a graph. However, some basic support is provided for simple
% cases.
%
% The idea is at follows: Graphs are specified hierarchically. For
% instance, consider the following graph specification:
%
% graph { a, b, c -> d -> {e -> f -> g, h} -> i, j -> k }
%
% Here, we have the *group* {e->f->g,h} inside the larger graph
% specification. Each group consists of sequence of *chains* like
% e->f->g or j->k.
%
% In order to facilitate the automatic positioning of nodes, the graph
% library will provide you with information about the position of
% nodes inside their groups and chains.
%
% As a chain is being parsed, a counter stored in
% /tikz/graphs/placement/element count is available that is advanced for
% each element in the chain.
%
% Additionally, a counter stored in placement/width is
% available. This "logical" width is defined recursively as follows: The
% width of a single node is computed by calling the key
% placement/logical node width, which should return a real or logical
% width of the node passed as a parameter in the macro \pgfmathresult. The
% width of a chain is the sum of the widths of its elements. The
% width of a  group is the maximum of the widths of its elements.
%
%
% Symmetrically, as a group is being constructed, a counter stored in
% placement/chain count is available that is advanced for each chain
% in the group. The number placement/depth is the defined
% recursively as follows: For a single node, the depth is
% computed by the key placement/logical node depth. The depth
% of a group is the sum of the depths of its elements. The depth of a
% chain is the maximum of the depth of its elements.
%
%
% The above keys get updated automatically. You should setup the key
% placement/compute position such that it uses the above keys to
% compute a good position for a new node based on the above
% keys. Typically, this key should execute node={shift=(...)} to setup
% the necessary shift for a new node.
%
% The key placement/compute position should not be called
% directly. Instead, the key placement/place should be used. This key
% has two effects: First, it calls placement/compute position. Second,
% it resets the length and normal counters. It will setup a completely
% new counting of lengths and counters inside the current scope.
%
% The placmenet/place key is executed automatically whenever a new
% node is automatically created. Furthermore, placement strategies
% will call this key.

\tikzgraphsset{
  placement/.cd,
  compute position/.code=\tikz@lib@graph@linear@pos,
  place/.code={%
    \pgfkeys{/tikz/graphs/placement/compute position}%
    \aftergroup\tikz@lib@graph@reset@locals%
    \pgfkeyssetvalue{/tikz/graphs/placement/element count}{0}%
    \pgfkeyssetvalue{/tikz/graphs/placement/chain count}{0}%
    \pgfkeyssetvalue{/tikz/graphs/placement/depth}{0}%
    \pgfkeyssetvalue{/tikz/graphs/placement/width}{0}%
    \pgfkeyssetvalue{/tikz/graphs/placement/local depth}{0}%
    \pgfkeyssetvalue{/tikz/graphs/placement/local width}{0}%
  },
  element count/.initial=0,
  chain count/.initial=0,
  depth/.initial=0,
  width/.initial=0,
  level/.initial=0,
  logical node depth/.code=\def\pgfmathresult{1},
  logical node width/.code=\def\pgfmathresult{1},
}%

\def\tikz@lib@graph@reset@locals{%
  \gdef\tikz@lib@graph@group@depth{0}%
  \gdef\tikz@lib@graph@group@width{0}%
}%


% Arrange nodes evenly
%
% This strategy works as follows: You specify a "chain shift vector"
% and a "group shift vector". Then each new element on a chain is
% shifted by the chain shift vector relative to the previous element
% on the chain. Similarly for each new element of a group.

\tikzgraphsset{
  Cartesian placement/.style={
    placement/place,
    placement/compute position/.code=\tikz@lib@graph@linear@pos%
  },
  chain shift/.initial={(1,0)},
  group shift/.initial={(0,-1)},
  grow right/.style={
    placement/place,
    chain shift={(#1,0)},
    @auto anchor horizontal=center,
    placement/logical node width/.code=\def\pgfmathresult{1}
  },
  grow right/.default=1,
  grow left/.style={
    placement/place,
    chain shift={(-#1,0)},
    @auto anchor horizontal=center,
    placement/logical node width/.code=\def\pgfmathresult{1}
  },
  grow left/.default=1,
  grow up/.style={
    placement/place,
    chain shift={(0,#1)},
    @auto anchor horizontal=center,
    placement/logical node width/.code=\def\pgfmathresult{1}
  },
  grow up/.default=1,
  grow down/.style={
    placement/place,
    chain shift={(0,-#1)},
    @auto anchor vertical=center,
    placement/logical node width/.code=\def\pgfmathresult{1}
  },
  grow down/.default=1,
  branch right/.style={
    placement/place,
    group shift={(#1,0)},
    @auto anchor horizontal=center,
    placement/logical node depth/.code=\def\pgfmathresult{1}
  },
  branch right/.default=1,
  branch left/.style={
    placement/place,
    group shift={(-#1,0)},
    @auto anchor horizontal=center,
    placement/logical node depth/.code=\def\pgfmathresult{1}
  },
  branch left/.default=1,
  branch up/.style={
    placement/place,
    group shift={(0,#1)},
    @auto anchor vertical=center,
    placement/logical node depth/.code=\def\pgfmathresult{1}
  },
  branch up/.default=1,
  branch down/.style={
    placement/place,
    group shift={(0,-#1)},
    @auto anchor vertical=center,
    placement/logical node depth/.code=\def\pgfmathresult{1}
  },
  branch down/.default=1,
  %
  % Sep shifts
  %
  grow right sep/.style={
    Cartesian placement,
    chain shift={(1pt,0)},
    @auto anchor horizontal=west,
    placement/logical node width/.code=\tikz@lib@graph@width@sep{##1}{#1}
  },
  grow right sep/.default=1em,
  grow left sep/.style={
    Cartesian placement,
    chain shift={(-1pt,0)},
    @auto anchor horizontal=east,
    placement/logical node width/.code=\tikz@lib@graph@width@sep{##1}{#1}
  },
  grow left sep/.default=1em,
  grow up sep/.style={
    Cartesian placement,
    chain shift={(0,1pt)},
    @auto anchor vertical=south,
    placement/logical node width/.code=\tikz@lib@graph@depth@sep{##1}{#1}
  },
  grow up sep/.default=1em,
  grow down sep/.style={
    Cartesian placement,
    chain shift={(0,-1pt)},
    @auto anchor vertical=north,
    placement/logical node width/.code=\tikz@lib@graph@depth@sep{##1}{#1}
  },
  grow down sep/.default=1em,
  grow right sep/.style={
    Cartesian placement,
    chain shift={(1pt,0)},
    @auto anchor horizontal=west,
    placement/logical node width/.code=\tikz@lib@graph@width@sep{##1}{#1}
  },
  %
  branch right sep/.style={
    Cartesian placement,
    group shift={(1pt,0)},
    @auto anchor horizontal=west,
    placement/logical node depth/.code=\tikz@lib@graph@width@sep{##1}{#1}
  },
  branch right sep/.default=1em,
  branch left sep/.style={
    Cartesian placement,
    group shift={(-1pt,0)},
    @auto anchor horizontal=east,
    placement/logical node depth/.code=\tikz@lib@graph@width@sep{##1}{#1}
  },
  branch left sep/.default=1em,
  branch up sep/.style={
    Cartesian placement,
    group shift={(0,1pt)},
    @auto anchor vertical=south,
    placement/logical node depth/.code=\tikz@lib@graph@depth@sep{##1}{#1}
  },
  branch up sep/.default=1em,
  branch down sep/.style={
    Cartesian placement,
    group shift={(0,-1pt)},
    @auto anchor vertical=north,
    placement/logical node depth/.code=\tikz@lib@graph@depth@sep{##1}{#1}
  },
  branch down sep/.default=1em,
  %
  @auto anchor horizontal/.style={
    nodes={anchor=\csname tikz@lib@graph@auto@\tikz@lib@graph@auto@h @\tikz@lib@graph@auto@v\endcsname},
    /utils/exec=\def\tikz@lib@graph@auto@h{#1}
  },
  @auto anchor vertical/.style={
    nodes={anchor=\csname tikz@lib@graph@auto@\tikz@lib@graph@auto@h @\tikz@lib@graph@auto@v\endcsname},
    /utils/exec=\def\tikz@lib@graph@auto@v{#1}
  },
  %
  %
  no placement/.style={
    placement/place,
    placement/compute position/.code=%
  }
}%

\def\tikz@lib@graph@auto@h{center}%
\def\tikz@lib@graph@auto@v{center}%

\def\tikz@lib@graph@auto@center@center{center}%
\def\tikz@lib@graph@auto@west@center{west}%
\def\tikz@lib@graph@auto@east@center{east}%
\def\tikz@lib@graph@auto@center@north{north}%
\def\tikz@lib@graph@auto@west@north{north west}%
\def\tikz@lib@graph@auto@east@north{north east}%
\def\tikz@lib@graph@auto@center@south{south}%
\def\tikz@lib@graph@auto@west@south{south west}%
\def\tikz@lib@graph@auto@east@south{south east}%

\def\tikz@lib@graph@linear@pos{%
  \pgfkeysgetvalue{/tikz/graphs/chain shift}\tikz@temp
  \expandafter\tikz@scan@one@point\expandafter\pgf@process\tikz@temp
  \pgf@process{\pgfpointscale{\pgfkeysvalueof{/tikz/graphs/placement/width}}{}}%
  \pgf@xa=\pgf@x%
  \pgf@ya=\pgf@y%
  \pgfkeysgetvalue{/tikz/graphs/group shift}\tikz@temp
  \expandafter\tikz@scan@one@point\expandafter\pgf@process\tikz@temp
  \pgf@process{\pgfpointscale{\pgfkeysvalueof{/tikz/graphs/placement/depth}}{}}%
  \advance\pgf@xa by\pgf@x%
  \advance\pgf@ya by\pgf@y%
  \edef\tikz@lib@graph@shift{(\the\pgf@xa,\the\pgf@ya)}
  \pgfkeys{/tikz/graphs/nodes/.expanded={shift={\tikz@lib@graph@shift}}}
}%


\def\tikz@lib@graph@width@sep#1#2{%
  \pgf@process{\pgfpointdiff{\pgfpointanchor{#1}{west}}{\pgfpointanchor{#1}{east}}}%
  \pgfmathparse{#2+\the\pgf@x}%
}%

\def\tikz@lib@graph@depth@sep#1#2{%
  \pgf@process{\pgfpointdiff{\pgfpointanchor{#1}{south}}{\pgfpointanchor{#1}{north}}}%
  \pgfmathparse{#2+\the\pgf@y}%
}%


% Circular arrangements
%
% This strategy works a bit like the arrange evenly strategy, but in
% polar coordinates. Both for the chains and the groups you specify a
% polar shift, which must be in the form "(delta degree:delta
% distance)". For each element in a chain, the delta degree is added
% to the chain degree, likewise for each element the delta distance is
% added. Similarly for groups.
%
% There is an initial degree and radius, stored in the key "phase" and
% "radius".

\tikzgraphsset{,
  circular placement/.style={
    placement/place,
    placement/compute position/.code=\tikz@lib@graph@circular@pos,%
    placement/logical node depth/.code=\def\pgfmathresult{1},
    placement/logical node width/.code=\def\pgfmathresult{1},
  },
  clockwise/.style={
    circular placement,
    group polar shift={(-360/#1:0)}
  },
  clockwise/.default=\tikzgraphVnum,
  counterclockwise/.style={
    circular placement,
    group polar shift={(360/#1:0)}
  },
  counterclockwise/.default=\tikzgraphVnum,
  chain polar shift/.initial={(0:1cm)},
  group polar shift/.initial={(60:0)},
  radius/.initial=1cm,
  phase/.initial=90,
}%

\def\tikz@lib@graph@circular@pos{%
  \pgfkeysgetvalue{/tikz/graphs/chain polar shift}\tikz@temp
  \expandafter\tikz@lib@graph@decompose@polar\tikz@temp%
  \pgf@process{\pgfpointscale{\pgfkeysvalueof{/tikz/graphs/placement/width}}{}}%
  \pgf@xa=\pgf@x%
  \pgf@ya=\pgf@y%
  \pgfkeysgetvalue{/tikz/graphs/group polar shift}\tikz@temp
  \expandafter\tikz@lib@graph@decompose@polar\tikz@temp%
  \pgf@process{\pgfpointscale{\pgfkeysvalueof{/tikz/graphs/placement/depth}}{}}%
  \advance\pgf@xa by\pgf@x%
  \advance\pgf@ya by\pgf@y%
  \pgfmathparse{\pgfkeysvalueof{/tikz/graphs/radius}}%
  \advance\pgf@ya by\pgfmathresult pt%
  \pgfmathsetmacro\tikz@temp{\the\pgf@xa+\pgfkeysvalueof{/tikz/graphs/phase}}%
  \edef\tikz@lib@graph@shift{(\tikz@temp:\the\pgf@ya)}
  \pgfkeys{/tikz/graphs/nodes/.expanded={shift={\tikz@lib@graph@shift}}}
}%


\def\tikz@lib@graph@decompose@polar(#1:#2){%
  \pgfmathsetlength\pgf@x{#1}%
  \pgfmathsetlength\pgf@y{#2}%
}%


% Grid arrangements
%
\tikzgraphsset{
  grid placement/.style={
    placement/place,
    placement/compute position/.code=\tikz@lib@graph@grid@pos,%
  },
}%


\def\tikz@lib@graph@grid@pos{%
  %
  % Make sure to set \tikzgraphwrapafter before we compute
  % the positions of nodes in the grid
  %
  \tikzgraphpreparewrapafter%
  %
  % find out the coordinate in the grid
  %
  \pgfkeysgetvalue{/tikz/graphs/placement/depth}\tikz@temp@depth%
  \pgfkeysgetvalue{/tikz/graphs/placement/width}\tikz@temp@width%
  \pgfmathsetmacro{\tikz@temp}{max(\tikz@temp@depth, \tikz@temp@width)}%
  \pgfmathsetmacro{\tikz@temp@col}{mod(\tikz@temp,\tikzgraphwrapafter)}%
  \pgfmathsetmacro{\tikz@temp@row}{div(\tikz@temp,\tikzgraphwrapafter)}%
  %
  % multiply row by the group shift and col by the chain shift
  %
  \pgfkeysgetvalue{/tikz/graphs/group shift}\tikz@temp%
  \expandafter\tikz@scan@one@point\expandafter\tikz@lib@graph@grid@pos@a\tikz@temp%
  \pgfkeysgetvalue{/tikz/graphs/chain shift}\tikz@temp%
  \expandafter\tikz@scan@one@point\expandafter\tikz@lib@graph@grid@pos@b\tikz@temp%
  %
  % apply the shift
  %
  \edef\tikz@lib@graph@shift{(\the\pgf@xa,\the\pgf@ya)}
  \pgfkeys{/tikz/graphs/nodes/.expanded={shift={\tikz@lib@graph@shift}}}
}%

\def\tikz@lib@graph@grid@pos@a#1{\pgf@process{\pgfpointscale{\tikz@temp@row}{#1}}\pgf@xa=\pgf@x\relax\pgf@ya=\pgf@y\relax}%
\def\tikz@lib@graph@grid@pos@b#1{\pgf@process{\pgfpointscale{\tikz@temp@col}{#1}}%
  \advance\pgf@xa by\pgf@x\relax
  \advance\pgf@ya by\pgf@y\relax}%

\tikzgraphsset{
  % Grids:
  wrap after/.initial=0,
  % Node sets:
  V/.code={%
    \def\tikzgraphV{#1}%
    \c@pgf@counta=0\foreach \tikz@dummy in {#1} {\global\advance\c@pgf@counta by1\relax}%
    \edef\tikzgraphVnum{\the\c@pgf@counta}
  },
  V={1},
  n/.style={V={1,...,#1},name shore V/.style={name=V}},
  W/.code={%
    \def\tikzgraphW{#1}%
    \c@pgf@counta=0\foreach \tikz@dummy in {#1} {\global\advance\c@pgf@counta by1\relax}%
    \edef\tikzgraphWnum{\the\c@pgf@counta}
  },
  W={1},
  m/.style={W={1,...,#1},name shore W/.style={name=W}},
  % Shores:
  name shore V/.style=,
  name shore W/.style=,
}%



%
% Quick graphs
%
% Parsing the graph syntax can take a lot of time for large
% graphs. This is mainly due to the transformation of commas and
% semicolons and all the internal bookkeeping.
%
% For large graphs, the quick graph syntax offers a greatly reduced
% syntax that will be processed much more quickly. This syntax is
% compatible with the normal graph syntax in the sense that all
% specifications written in the reduced syntax are also legal in the
% normal syntax, but not the other way round.
%
% The following rules apply:
%
% 1) A quick graph consists of a sequence of either nodes, edges sequences, or
% groups. These are separated by commas or semicolons.
%
% 2) Every node is of the form
%
% "node name"/"node text"[options]
%
% Here, the quotation marks are mandatory. The part '/"node text"' may
% be missing, in which case the node name is used as the node
% text. The options may also be missing. The node name may not contain
% any "funny" characters (unlike in the normal graph command).
%
% 3) Every edge is of the form
%
% <node spec> <connector> <node spec> <connector> <node spec> ... <connector> <node spec>;
%
% Here, the <node spec> are node specifications as described above,
% the <connector> is one of the four connectors "->", "<-", "--", and
% "<->", followed by optional options in square
% brackets. The semicolon may be replaced by a comma.
%
% 4) Every group is of the form
%
% { [options] lines };
%
% The options are compulsory, the lines must be a sequence of nodes,
% edges, and groups as described above. The semicolon can, again, be
% replaced by a comma.
%
%
% Things that are allowed in the normal syntax, but not in the quick
% syntax, include:
%
% - Connecting a node and a group as in a->{b,c}.
% - Node names without quotation marks.
% - Using subgraphs (graph macros).
% - Using graph sets.
% - Using graph color classes.
% - Using anonymous nodes.
% - Using the "fresh nodes" option.
% - Using sublayouts.
% - Using the -!- edge connector.
% - Using the "simple" option.
% - Using edge annotations.
%
% Example:
%
% \graph [quick] {
%   "a"/"$a$"[red] -> "b"[blue] -> "c";
%   "b"            -> "d";
%   "c"            ->[thick] "e";
%   { [same layer] "c", "d" };
% };
%
%

\tikzgraphsset{quick/.is if=tikz@graph@quick}%
\newif\iftikz@graph@quick

\def\tikz@lib@graphs@parse@quick@graph{
  % Ok, we are before the initial brace. We open a group and start
  % setting up some bookkeeping
  \begingroup%
  \global\tikz@qnode@count0\relax%
  \let\tikzgraphnodepath\pgfutil@empty%
  \tikz@q@outertrue%
  \afterassignment\tikz@lib@graphs@quick@main\let\pgf@temp=%
}%

\newcount\tikz@qnode@count
\newif\iftikz@q@outer

\def\tikz@lib@graphs@quick@main{%
  \afterassignment\tikz@lib@graphs@quick@handle\let\pgf@let@token=%
}%
\def\tikz@lib@graphs@quick@handle{%
  \ifx\pgf@let@token"%
    \expandafter\tikz@lib@graphs@quick@first@node%
  \else%
    \expandafter\tikz@lib@graphs@quick@other%
  \fi%
}%
\def\tikz@lib@graphs@quick@other{%
  \let\tikz@next\tikz@lib@graphs@quick@error%
  \ifx\pgf@let@token\egroup%
    \let\tikz@next\tikz@lib@graphs@quick@end@group%
  \else%
    \ifx\pgf@let@token\bgroup%
      \let\tikz@next\tikz@lib@graphs@quick@start@group%
    \else%
      \ifx\pgf@let@token\par%
        \let\tikz@next\tikz@lib@graphs@quick@main%
      \fi%
    \fi
  \fi%
  \tikz@next%
}%

\def\tikz@lib@graphs@quick@error#1{\tikzerror{Unexpected token '\string#1' in quick graph syntax}\tikz@lib@graphs@quick@main}%
\def\tikz@lib@graphs@quick@start@group{%
  \pgfutil@ifnextchar[\tikz@lib@graphs@quick@start@group@{\tikzerror{Group
      in quick graph syntax must start with options.}}
}%

\def\tikz@lib@graphs@quick@start@group@[#1]{%
  \begingroup%
    \tikz@q@outerfalse%
    \tikzgraphsset{#1}%
    \tikz@lib@graphs@quick@main%
}%
\def\tikz@lib@graphs@quick@end@group{%
  \iftikz@q@outer%
    \endgroup%
    % Ok, cleanup!
    \pgfutil@loop%
    \ifnum\tikz@qnode@count>0\relax%
      \expandafter\global\expandafter\let\csname tikz@gr@q@@\csname tikz@gr@qn@@\the\tikz@qnode@count\endcsname\endcsname\relax%
      \expandafter\global\expandafter\let\csname tikz@gr@qn@@\the\tikz@qnode@count\endcsname\relax%
      \global\advance\tikz@qnode@count by-1\relax%
    \pgfutil@repeat%
    \expandafter\tikz@lib@graph@main@done%
  \else%
    \endgroup%
    \expandafter\tikz@lib@graphs@end@group@%
  \fi%
}%
\def\tikz@lib@graphs@end@group@{%
  \pgfutil@ifnextchar;{\expandafter\tikz@lib@graphs@quick@main\pgfutil@gobble}{%
    \pgfutil@ifnextchar,{\expandafter\tikz@lib@graphs@quick@main\pgfutil@gobble}{%
      \tikzerror{Graph groups in quick syntax must be followed by a semicolon or a comma.}%
    }%
  }%
}%


\def\tikz@lib@graphs@quick@first@node{%
  \let\tikz@quick@prev@node\relax%
  \tikz@lib@graphs@quick@node%
}%

\def\tikz@lib@graphs@quick@node#1"{%
  \def\tikzgraphnodename{#1}%
  \let\tikzgraphnodetext\tikzgraphnodename%
  \iftikzgraphsautonumbernodes
    \edef\tikzgraphnodename{\tikzgraphnodename\tikz@lib@auto@sep\the\tikz@lib@auto@number}%
    \global\advance\tikz@lib@auto@number by1\relax%
  \fi
  \pgfutil@ifnextchar/\tikz@lib@graphs@quick@text\tikz@lib@graphs@quick@opt%
}%
\def\tikz@lib@graphs@quick@text/"#1"{%
  \def\tikzgraphnodetext{#1}%
  \tikz@lib@graphs@quick@opt%
}%
\def\tikz@lib@graphs@quick@opt{%
  \pgfutil@ifnextchar[\tikz@lib@graphs@quick@withopt{\tikz@lib@graphs@quick@withopt[]}%]
}%
\def\tikz@lib@graphs@quick@withopt[#1]{%
  % Test, whether node already exists
  \expandafter\ifx\csname tikz@gr@q@@\tikzgraphnodename\endcsname\relax%
    % New node
    \global\advance\tikz@qnode@count by1\relax%
    \expandafter\global\expandafter\let\csname tikz@gr@q@@\tikzgraphnodename\endcsname\pgfutil@empty% not relax
    \expandafter\global\expandafter\let\csname tikz@gr@qn@@\the\tikz@qnode@count\endcsname\tikzgraphnodename% back pointer
    \let\tikzgraphnodefullname\tikzgraphnodename%
    \let\tikzgraphnodeas\tikzgraphnodeas@default%
    \node [%
      name=\tikzgraphnodename,%
      graphs/redirect unknown to tikz,
      /tikz/graphs/.cd,%
      /tikz/graphs/@nodes styling,%
      #1]%
      {%
        \tikzgraphnodeas%
      };%
  \else%
    %
    % Handle late options and operators
    \tikzgdlatenodeoptionacallback{\tikzgraphnodename}%
    \node also[graphs/redirect unknown to tikz,/tikz/graphs/.cd,#1](\tikzgraphnodename);%
  \fi%
  % Connect, if necessary
  \tikz@lig@graph@quikc@make@edge@if@necessary%
  \tikz@lib@graphs@quick@scan@after@node%
}%

\def\tikz@lig@graph@quikc@make@edge@if@necessary{%
  \ifx\tikz@quick@prev@node\relax%
  \else%
    \tikz@lib@graphs@quick@make@edge%
  \fi%
}%

\def\tikz@lib@graphs@quick@scan@after@node{%
  \pgfutil@ifnextchar,\tikz@lib@graphs@quick@comma{%
    \pgfutil@ifnextchar;\tikz@lib@graphs@quick@semi{%
      \pgfutil@ifnextchar\egroup{\tikz@lib@graphs@quick@semi;}{%
        \pgfutil@ifnextchar\par{\expandafter\tikz@lib@graphs@quick@scan@after@node\tikz@lib@graphs@quick@gobble@par}%
        \tikz@lib@graphs@quick@connector}}}%
}%
\long\def\tikz@lib@graphs@quick@gobble@par#1{}%

\def\tikz@lib@graphs@quick@comma,{\tikz@lib@graphs@quick@main}%
\def\tikz@lib@graphs@quick@semi;{\tikz@lib@graphs@quick@main}%
\def\tikz@lib@graphs@quick@connector#1#2{%
  \def\tikz@lib@graphs@quick@edge@kind{#1#2}%
  \pgfutil@ifnextchar>\tikz@lib@graphs@back@edge{%
    \pgfutil@ifnextchar[\tikz@lib@graphs@quick@connector@handle@opt{\tikz@lib@graphs@quick@connector@handle@opt[]}%]
  }%
}%
\def\tikz@lib@graphs@back@edge#1{%
  \expandafter\def\expandafter\tikz@lib@graphs@quick@edge@kind\expandafter{\tikz@lib@graphs@quick@edge@kind#1}%
  \pgfutil@ifnextchar[\tikz@lib@graphs@quick@connector@handle@opt{\tikz@lib@graphs@quick@connector@handle@opt[]}%]
}%
\def\tikz@lib@graphs@quick@connector@handle@opt[#1]{%
  \def\tikz@lib@graphs@quick@edge@options{#1}%
  \let\tikz@quick@prev@node\tikzgraphnodename%
  \tikz@lib@graphs@quick@scan@after@connector%
}%
\def\tikz@lib@graphs@quick@scan@after@connector{%
  \pgfutil@ifnextchar\par{\expandafter\tikz@lib@graphs@quick@scan@after@connector\tikz@lib@graphs@quick@gobble@par}{%
    \pgfutil@ifnextchar"{\expandafter\tikz@lib@graphs@quick@node\pgfutil@gobble}{%
      \tikzerror{Quotation marks expected after edge connector}%
      }%
  }%
}%

\def\tikz@lib@graphs@quick@make@edge{%
  {
    \expandafter\tikz@lib@graphs@quick@make@edge@styling\expandafter{\tikz@lib@graphs@quick@edge@options}%
    \expandafter\expandafter\expandafter\tikz@lib@graphs@quick@make@edge@for%
    \expandafter\expandafter\expandafter{\expandafter\tikz@quick@prev@node\expandafter}\expandafter{\tikzgraphnodename}%
  }
}%

\def\tikz@lib@graphs@quick@make@edge@styling#1{%
  \tikz@enable@edge@quotes%
  \tikzgraphsset{.unknown/.code=\tikz@lib@graph@unknown@edge@option{##1},#1}%
}%

\def\tikz@lib@graphs@quick@make@edge@for#1#2{%
  \pgfkeysgetvalue{/tikz/graphs/@edges styling}\pgf@tempa
  \pgfkeysgetvalue{/tikz/graphs/@edges node}\pgf@temp@b
  \expandafter\expandafter\expandafter\tikz@lib@graphs@quick@make@edge@for@with%
  \expandafter\expandafter\expandafter{\expandafter\pgf@tempa\expandafter}\expandafter{\pgf@temp@b}{#1}{#2}%
}%

\def\tikz@lib@graphs@quick@make@edge@for@with#1#2#3#4{%
  \tikzgraphsset{new \tikz@lib@graphs@quick@edge@kind={#3}{#4}{#1}{#2}}%
}%

\endinput
