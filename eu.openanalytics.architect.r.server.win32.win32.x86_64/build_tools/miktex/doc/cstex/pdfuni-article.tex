
\mubyte \edieresis ^^c3^^ab\endmubyte
\def\edieresis{\"e}
\mubyte \Edieresis ^^c3^^8b\endmubyte
\def\Edieresis{\"E}
\mubyte \ntilde ^^c3^^b1\endmubyte % mapujeme ñ na \ntilde
\def\ntilde{\~n}                   % definujeme \ntilde pro běžný tisk

\parindent=12pt

\chyph
\input opmac \input pdfuni

\hyperlinks\Blue\Green  \outlines1

\activettchar"
\catcode`<=13
\def<#1>{\hbox{$\langle$\it#1\/$\rangle$}}

%\pdfunidef\tmp{\dots § t\~e_~x^\%\$tí\&č\#e\_k}
%\insertoutline{\tmp}

\tit PDFuni -- akcenty v PDF záložkách

\centerline{\it Petr Olšák}\bigskip

Záložky v PDF dokumentu (outlines, bookmarks) jsou texty, které za jistých okolností
zobrazují PDF prohlížeče a které jsou klikací: rozvírají se a nabízejí
stromovou strukturu. A hlavně klik na text v záložce způsobí skok na tomu
odpovídající místo v dokumentu. Záložky se nikdy netisknou. Mnozí uživatelé
pdf\TeX{}u si jistě všimli, že český nebo slovenský text má někdy sklon ke
znehodnocení akcentovaných znaků, pokud jej posíláme do záložek. V
tomto článku si vysvětlíme pozadí těchto problémů a popíšeme makro PDFuni 
pro plain\TeX{}, které problém s českými a slovenskými texty v záložkách a
dalších podobných místech řeší.

\sec Kódovací standardy pro PDF stringy

PDF stringy jsou jednak texty v záložkách a dále texty v informaci o PDF
dokumentu vložené příkazem "\pdfinfo" o autorovi, názvu díla, klíčových
slovech a podobně. Údaje z "\pdfinfo" lze z dokumentu vydolovat programem
"pdfinfo" na příkazovém řádku a některé PDF prohlížeče je také dokáží
zobrazit.

PDF stringy nejsou součástí hlavního textu dokumentu, takže pro ně není
použit font vybraný autorem dokumentu. Místo toho si PDF prohlížeč k zobrazení
PDF stringů zavolá font, který je zrovna v době prohlížení po ruce a který
mu nabídne operační systém.

Specifikace PDF formátu \cite[pdfstandard] vymezuje (v dodatku D) pro PDF
stringy dvě možná kódování, První je jednobytové (tj. jeden znak je
reprezentován jedním bytem) a jmenuje se "PDFDocEncoding". Toto kódování se
shoduje s viditelnými znaky z ISO-8859-1, takže západoevropané s ním nemají problémy.
Nevýhodou tohoto kódování je, že v něm nenajdeme většinu akcentovaných znaků
české a slovenské abecedy. Nestaráme-li se o nic jiného, je uložený string
interpetován PDF prohlížečem jako kódovaný podle "PDFDocEncoding", takže 
nám to zákonitě poničí české texty.

Kvůli tomu jsem se rozhodl do makra OPmac \cite[opmac] zařadit konverzní
proces, který veškeré české a slovenské akcentované znaky převede do jejich
ASCII podobných znaků bez akcentu a teprve takto převedený text uloží jako
PDF string. V záložkách pak sice nevidíme akcenty, ale taky tam nevidíme
nesmysly, ke kterým to je náchylné. I kdyby byly PDF stringy správně
kódované pro češtinu (jak popíšeme za chvíli), může se stát, že nejsou
správně v prohlížeči zobrazeny, protože správné zobrazení navíc závisí na
správném nastavení locales a na dalších systémových parametrech. Rozhodl jsem
se tedy pro robustní řešení: PDF stringy generované makrem OPmac 
budou jenom v ASCII. A nehodlám to měnit.

Nicméně uživatelé např. balíčku "hyperref" si jistě všimli, že za jistých
příznivých okolností konstelace hvězd (tj. "hyperref" pracuje v UNICODE a
PDF prohlížeč má správně nastaveny veškeré systémové parametry) dokáží
vygenerovat nezmršené české texty i v záložkách. Je to tím, že PDF specifikace
nabízí druhou (a poslední) možnost kódování PDF stringů, kterou nazývá 
"UTF-16BE Unicode" (dále jen UNICODE). 
Kódování PDF stringu v tomto režimu se pozná tak, že první dva
byty stringu mají hodnotu 254, 255. Je to prefix pro přepnutí do tohoto
kódování. Podrobněji viz kapitolu 3.8.1 (String Types) 
v~manuálu~\cite[pdfstandard].

Nechceme-li ukládat do PDF stringu jednotlivé byty nativně, máme možnost je
popsat pomocí backslashe následovaném třemi oktalovými číslicemi vyjadřující
hodnotu bytu. String \uv{Cvičení je zátěž} pak může v PDF stringu
být zapsaný takto:

\begtt
\376\377\000C\000v\000i\001\015\000e\000n\000\355\000\040\000j\000e\000\040
\000z\000\341\000t\001\033\001\176
\endtt

V uvedeném příkladu vidíme nejprve byty 254, 255 zapsané oktalově, dále je
oktalově zapsaná nula následovaná písmenem C (tyto dva byty reprezentují
písmeno C v UTF-16 kódování). Podobně jistě dokážete dešifrovat písmena
\uv{v} a \uv{i}. Dále "\001\015" je 01,0D hexadecimálně a to reprezentuje
znak \uv{č} v UNICODE. Dál si ten string dočtěte sami...

Právě takové stringy musíme generovat \TeX{}em, pokud chceme vidět v
záložkách správně napsané české a slovenské znaky. 
Je asi nemyslitelné, abychom psali ručně do dokumentu
něco jako:

\begtt
\pdfoutline goto name{cviceni} count0 {% záložka
\string\376\string\377\string\000C\string\000v\string\000i\string\001\string\015%
\string\000e\string\000n\string\000\string\355\string\000\string\040\string\000j%
\string\000e\string\000\string\040\string\000z\string\000\string\341%
\string\000t\string\001\string\033\string\001\string\176}%
\pdfdest name{cviceni} xyz\relax % cíl odkazu
\endtt

Sice je tímto způsobem na primitivní úrovni vytvořena záložka s textem
\uv{Cvičení je zátěž}, ale kdybychom to takto dělali pořád, byla by to
opravdu zátěž.
Proto jsem vytvořil makro PDFuni, které uživatelům \csplain{}u 
ulehčí práci. Uživatelé \LaTeX{}u nic takového nepotřebují, protože mají
balíček "hyperref".


\sec Použití balíčku PDFuni

Protože OPmac generuje do záložek texty bez akcentů, je potřeba zavolat
PDFuni až po "\input opmac". Balíček PDFuni předefinuje implicitní 
konvertor z OPmac pro výstup do PDF stringů (konvertor do ASCII) na nový konvertor 
do UNICODE. Takže stačí na začátku dokumentu psát:

\begtt
\input opmac
\input pdfuni
\endtt

Od této chvíle příkaz "\outlines<num>" generuje do záložek správně české a
slovenské texty. Využívá přitom automaticky generovaný obsah. OPmac nabízí
ještě příkaz "\insertoutline{<text>}", který vloží do záložek jeden údaj.
Tento příkaz vkládá "<text>" tak jak je bez konverze, z čehož plyne, že tam
české akcenty nebudou jednoduše fungovat. Je potřeba použít příkaz
"\pdfunidef\makro{<text>}", který je definován v balíčku PDFuni a který vloží
do "\makra" zkonvertovaný "<text>" v kódování UNICODE s~oktalovými přepisy.
Kategorie backslashe tam je 12 (obyčejný znak). Takže je potřeba psát:

\begtt
\pdfunidef\tmp{Tady je český text}
\insertoutline{\tmp}
\endtt

PDF dokument může obsahovat PDF stringy různě kódované. Takže funguje
například i toto:

\begtt
    % tento PDF string je v UNICODE:
\pdfunidef\tmp{Čeština je dřina}\insertoutline{\tmp} 
    % tento PDF string je v PDFDocEncoding:
\insertoutline{Jan Hus vnesl do jazyka velkou hloupost: zavedl akcenty.}
\endtt

Příkaz pdf\TeX{}u "\pdfinfo" zanáší do PDF dokumentu doplňující informace,
například takto:

\begtt
\pdfinfo {/Author (Autor)
          /Title (Tutul)
          /Creator (csplain + OPmac)
          /Subject (clanek)
          /Keywords (klicova slova)}
\endtt

Je potřeba vědět, že PDF stringem je zde každý údaj uzavřený do závorky,
přičemž některé tyto PDF stringy mohou být kódovány podle PDFDocEncoding a
jiné podle UNICODE. Takže je možné údaje zanést třeba takto:

\begtt
\def\author{Petr Olšák}\pdfunidef\author{\author}
\def\title{PDFuni - akcenty v PDF záložkách}\pdfunidef\title{\title}
\def\subject{článek}\pdfunidef\subject{\subject}
\pdfinfo {/Author (\author)
          /Title (\title)
          /Creator (csplain + OPmac)
          /CreationDate (D:20130517000000)
          /Subject (\subject)
          /Keywords (TeX; pdfTeX; PDF; OPmac)}
\endtt

\def\author{Petr Olšák}\pdfunidef\author{\author}
\def\title{PDFuni - akcenty v PDF záložkách}\pdfunidef\title{\title}
\def\subject{článek}\pdfunidef\subject{\subject}
\pdfinfo {/Author (\author)
          /Title (\title)
          /Creator (csplain + OPmac)
          /CreationDate (D:20130517000000)
          /Subject (\subject)
          /Keywords (TeX; pdfTeX; PDF; OPmac)}

Poznamenejme, že další informace "/Producer" a "/ModDate" vloží většinou
nejlépe pdf\TeX{} sám, nicméně uživatel je rovněž může předefinovat. Jsou-li oba
údaje "/CreationDate" a "/ModDate" nevyplněny, pak oba obsahují čas
vzniku PDF souboru. Jak je možné si všimnout,
chcete-li "/CreationDate" vyplnit podle svého, je nutné dodržet
formát "D:YYYYMMDDhhmmss".


\sec Možnost rozšíření kódovací tabulky o další znaky

Příkaz "\pdfunidef" konvertuje správně všechny viditelné ASCII znaky s
kategoriemi 11, 12 (písmena, znaky) a dále znaky s kategoriemi 7, 8 ("^", "_"). 
Ostatní viditelné ASCII znaky jiných kategorií (např. "$",~"&") jsou ignorovány.
Chcete-li do PDF stringu dopravit znak jiné
kategorie než 7, 8, 11, 12, musí mít takový znak 
před sebou backslash, tedy "\\", "\{", "\}", "\$", "\&", "\#", "\~", "\%".

Dále "\pdfunidef" konvertuje znaky kategorie 11 nebo 12, které nejsou ASCII,
pokud jsou zahrnuty do seznamu "\pdfunichars". Tento seznam implicitně
obsahuje znaky Áá Ää Čč Ďď Éé Ěě Íí Ĺĺ Ľľ Ňň Óó Öö Ôô Ŕŕ Řř Šš Ťť Úú Ůů Üü
Ýý Žž. Oktalový zápis těchto znaků (tedy výstup konverze) ve stejném pořadí
je uložen v seznamu "\pdfunicodes". Kódy jsou tam zapisovány bez backslashů
a bez první nuly a jsou odděleny od sebe příkazem "\or". Podívejte se do
souboru "pdfuni.tex" nebo do následující sekce, jak vypadají implicitní
hodnoty těchto seznamů. Chcete-li rozšířit tuto konverzní tabulku, je možné
použít příkaz "\addto" z OPmac například takto:

\begtt
\addto\pdfunichars{Ëë}  \addto\pdfinicodes{\or00313\or00353}
\endtt

Uvedený příklad rozšiřuje seznam "\pdfunichars" o znaky Ë, ë a dále
rozšiřuje seznam "\pdfunicodes" o kódy "\000\313" a "\000\353". Na tyto kódy
se budou znaky Ë, ë (v tomto pořadí) konvertovat.\fnote
{Aby uvedený příklad fungoval, musí mít znaky Ë, ë svou reprezentaci ve
vnitřním kódování \TeX{}u. To je u 16bitových \TeX{}ových mašin zařízeno
automaticky. Při použití \csplain{}u a enc\TeX{}u je možné použít třeba
{\tt\bslash input t1code}.}

Příkaz "\pdfunidef" ještě před započetím konverze provede úplnou expanzi
svého parametru "<text>". Před touto expanzí je vhodné předefinovat některá
makra, např. "\def\TeX{TeX}". Taková činnost je zanesena do seznamu
"\pdfunipre", který je (uvnitř skupiny) předřazen před expanzí parametru
"<text>". Kromě speciálních expanzí maker tam jsou povelem
"\let\makro\relax" zabezpečena makra, která expandovat v dané chvíli
nechceme. Jsou to makra "\ae", "\P" atd., která posléze 
vstoupí do konverze a promění se na svůj oktalový kód.
Přidělení tohoto kódu se děje v seznamu "\pdfunipost" pomocí příkazu 
"\odef\makro<6ciferný kód><mezera>". Např. příkazem 
"\odef\ae000346" říkáme, že makro "\ae" se zkonvertuje na "\000\346".
Čtenáři doporučuji podívat se na obsahy seznamů "\pdfunipre" a "\pdfunipost"
do souboru "pdfuni.tex" nebo do následující sekce.

Chcete-li rozšířit konverzní tabulku týkající se maker, lze to udělat
pomocí "\addto" například takto:

\begtt
\addto\pdfunipre{\let\endash\relax} \addto\pdfunipost{\odef\endash040023 }
\endtt

Tento příklad přidává do tabulky údaj o konverzi makra "\endash". Ve fázi
expandování parametru je jeho expanze potlačena a dále se zkonvertuje na
kód "\040\023", což je dle UNICODE pomlka na půlčtverčík.

Kontrolní sekvence, které nejsou během expanze parametru "<text>"
expandovány a nemají deklarován svůj kód pomocí "\odef", jsou při konverzi 
ignorovány.

Nyní už čtenář patrně ví, proč při použití \csplain{}u s enc\TeX{}em funguje
správně i právnický znak~§, ačkoli není uveden v seznamu "\pdfunichars".
Enc\TeX{} jej totiž mapuje na kontrolní sekvenci "\S", takže kdykoli
napíšeme~§, promění se to automaticky v "\S". 
Znak "\S" je ošetřen v seznamu "\pdfunipre", kde je napsáno
"\let\S\relax". A v seznamu "\pdfunipost" je definice "\odef\S000247".


\sec Popis interních maker a s nimi spojených triků

Makra z "pdfuni.tex" obsahují několik zajímavých triků. Jejich popis je
určen pro pokročilé \TeX{}isty.

Soubor "pdfuni.tex" je kódován v UTF-8 kódování. Je nutné, aby byl přečten
enc\TeX{}em nebo 16bitovou mašinou, protože chceme s jednotlivými
akcentovanými znaky v \TeX{}u pracovat jako s jedním tokenem. Proto je na
začátku souboru test, zda je použita správná \TeX{}ová mašina:

\begtt
\def\tmp#1#2\end{\if$#2$\else 
\errmessage {This file is UTF-8 encoded. Use TeX+encTeX or 16bit TeX engine.}\fi}%
\tmp č\end
\endtt

Dále definujeme na začátku souboru makro "\Bslash", které expanduje na
backslash kategorie 12. Totéž dělá OPmac s makrem "\bslash", ale OPmac
hodnotu tohoto makra v různých místech mění. Zde potřebujeme mít jistotu, že
to bude pořád backslash. Kromě toho potřebujeme pracovat s pomocným čítačem
"\tmpnum". Protože nevíme, zda je zavolán OPmac (makro PDFuni funguje i bez
něj), je za předpokladu nedefinovanosti "\newcount" tento čítač deklarován.

\begtt
\ifx\tmpnum\undefined \csname newcount\endcsname\tmpnum \fi
{\lccode`\?=`\\  \lowercase{\gdef\Bslash{?}}}
\endtt

Následuje výpis implicitní hodnoty seznamů "\pdfunichars" a
"\pdfunicodes"

\begtt
\edef\pdfunichars{\Bslash()
   ÁáÄäČčĎď ÉéĚěÍíĹĺ ĽľŇňÓóÖö ÔôŔŕŘřŠš ŤťÚúŮůÜü ÝýŽž
}
\def\pdfunicodes {\or00134\or00050\or00051\or                       % \ ()
   00301\or00341\or00304\or00344\or01014\or01015\or01016\or01017\or % ÁáÄäČčĎď
   00311\or00351\or01032\or01033\or00315\or00355\or01071\or01072\or % ÉéĚěÍíĹĺ
   01075\or01076\or01107\or01110\or00323\or00363\or00326\or00366\or % ĽľŇňÓóÖö
   00324\or00364\or01124\or01125\or01130\or01131\or01140\or01141\or % ÔôŔŕŘřŠš  
   01144\or01145\or00332\or00372\or01156\or01157\or00334\or00374\or % ŤťÚúŮůÜü
   00335\or00375\or01175\or01176%                                   % ÝýŽž
}
\endtt

Je vidět, že do seznamu jsou zařazeny i hodnoty "\", "(", ")". Je to tím, že
tyto znaky nechceme konvertovat do podoby "\000<znak>", protože by v
PDF stringu zlobily (backslash je escape prefix a kulaté závorky ohraničují
string). Je tedy bezpečnější je do PDF stringu zapsat v úplné oktalové notaci.

Pokračuje výpis implicitní hodnoty "\pdfunipre" a "\pdfunipost" a
některých speciálních kódů.

\begtt
\def\pdfunipre {\def\TeX{TeX}\def\LaTeX{LaTeX}\def~{ }\def\ { }%
   \let\ss\relax \let\l\relax \let\L\relax 
   \let\ae\relax \let\oe\relax \let\AE\relax \let\OE\relax 
   \let\o\relax \let\O\relax \let\i\relax \let\j\relax \let\aa\relax
   \let\AA\relax \let\S\relax \let\P\relax \let\copyright\relax
   \let\dots\relax \let\dag\relax \let\ddag\relax 
   \let\clqq\relax \let\crqq\relax \let\flqq\relax \let\frqq\relax
   \let\promile\relax \let\euro\relax
   \def\\{\Bslash}\let\{\relax \let\}\relax
   \def\${\string$}\def\&{\string&}\def\_{\string_}\def\~{\string~}% 
}
\def\pdfunipost {\odef\ss000337 \odef\l001102 \odef\L001101 
   \odef\AE000306 \odef\ae000346 \odef\OE001122 \odef\oe001123 
   \odef\o000370 \odef\O000330 \odef\aa000345 \odef\AA000305
   \odef\i001061 \odef\j002067  \odef\S000247 \odef\P000266
   \odef\copyright000251 \odef\dots040046 \odef\dag040040 \odef\ddag040041
   \odef\clqq040033 \odef\crqq040034 \odef\flqq000253 \odef\frqq000273
   \odef\promile040060 \odef\euro040254 \odef\{000173 \odef\}000175
   \odef\#000043
}
\def\pdfinitcode{\Bslash376\Bslash377}
\def\pdfspacecode{\Bslash000\Bslash040}
\def\pdfcircumflexcode{\Bslash000\Bslash136}
\def\pdflowlinecode{\Bslash000\Bslash137}
\endtt
\relax\let\tmp=$  % Editor mi obarvuje matematický mód i když nechci.

Nyní přistoupíme na to hlavní, na definici makra "\pdfunidef<makro>{<text>}".

{\catcode`\<=12
\begtt
\def\pdfunidef#1#2{\bgroup \def\tmpa{\noexpand#1}%
   \pdfunipre \edef\tmp{#2}\edef\out{\pdfinitcode}%
   \def\odef##1##2##3##4##5##6##7 
      {\def##1{&\edef\out{\out\Bslash##2##3##4\Bslash##5##6##7}}}\pdfunipost
   \def\insertbslashes ##1##2##3##4##5{\Bslash0##1##2\Bslash##3##4##5}%
   \def\gobbletwowords ##1 ##2 {}%
   \tmpnum=0 
   \loop 
      \sfcode\tmpnum=0 \advance\tmpnum by1
      \ifnum \tmpnum<128 \repeat   
   \tmpnum=0 
   \expandafter \pdfunidefA \pdfunichars {}%
}
\endtt
\relax}

"\pdfunidef" nejprve zahájí 
skupinu a zapamatuje si jméno "<makra>", které
má definovat, do "\tmpa". Dále spustí "\pdfunipre" a expanduje parametr "<text>".
Do pracovního makra "\out" bude postupně střádat výsledek konverze. Výchozí hodnota je
"\pdfinitcode", což je "\376\377", neboli prefix UNICODE stringu.
Dále je definováno pomocné makro "\odef<makro><6 oktalových cifer><mezera>"
poněkud trikoidním způsobem. Makro provede "\def<makro>{&<činnost>}".
Proč tam je ten znak "&" bude vysvětleno později. Činností, kterou makro
provede, je přidání odpovídajících oktalových cifer do výstupního makra
"\out", přitom k těmto cifrám jsou přidány backslashe. Seznam "\pdfunipost" obsahuje
jednotlivé příkazy "\odef", tedy připraví makra ke konverzi. 

Dále je připraveno pomocné makro "\insertbslashes<5 oktalových cifer>",
které přidá šestou nulu na začátek a vloží na správné místo backslashe.
Makro "\gobbletwowords" odstraní dvě následující slova ukončená mezerou.
Budeme to potřebovat za chvíli.

V závěru této části je cyklem uloženo každému znaku s kódem menším než 128
sfkód rovný nule. Podle této hodnoty sfkódu poznáme, že znak je z dolní
ASCII tabulky a bude kódován na "\000<znak>". Ostatní znaky vyjmenované v
seznamu "\pdfunichars" budou mít každý svůj sfkód postupně se zvětšující od
jedničky. To udělá makro "\pdfunidefA", které postupně ze seznamu
"\pdfunichars" odlupuje jednotlivé znaky a dává jim odpovídající sfkódy.

\begtt
\def\pdfunidefA #1{\if\relax#1\relax%
      \def\next{\let\next= }\afterassignment\pdfunidefB \expandafter\next\tmp\end
   \else \advance\tmpnum by1 \sfcode`#1=\tmpnum \expandafter \pdfunidefA
   \fi
}
\endtt

Postupné odlupování končí, když je načten prázdný parametr (poznáme to
podle "\if\relax#1\relax", to se totiž sejde první "\relax" s druhým).
V takovém případě zahájíme postupné odlupování expandovaného "<text>", který máme
v "\tmp". Konec druhého řádku ukázky po provedení "\expandafter" a následné
expanzi "\next" vypadá takto:

\begtt
\let\next= <expandovaný text>\end
\endtt

Mezera za rovnítkem je důležitá, protože další případná mezera se pak
skutečně přiřadí. Příkaz "\let\next" odloupne z "<expandovaného textu>" 
první token a po přiřazení spustí "\pdfunidefB".

\begtt
\def\pdfunidefB {%
   \ifcat x\noexpand\next \pdfunidefC
   \else \ifcat.\noexpand\next \pdfunidefC
   \else \ifcat\noexpand\next\space \edef\out{\out\pdfspacecode}%
   \else \ifcat^\noexpand\next \edef\out{\out\pdfcircumflexcode}%
   \else \ifcat_\noexpand\next \edef\out{\out\pdflowlinecode}%
   \else \expandafter\ifx\expandafter&\next
   \fi\fi\fi\fi\fi\fi
   \ifx\next\end \edef\next{\def\tmpa{\out}}\expandafter\egroup\next \else
      \def\next{\let\next= }\afterassignment\pdfunidefB \expandafter\next
   \fi
}
\endtt

Makro "\pdfunidefB" prošetří obsah "\next" a podle jeho typu provede jeden
krok konverze. V závěru své činnosti zavolá rekurzivně samo sebe a odloupne
z "<expandovaného textu>" další token. To dělá pořád, dokud nenarazí na "\end".
V případě, že narazilo na "\end", pak do "\next" vloží:

\begtt
\def<makro>{<zkonvertovaný výstup>}
\endtt
%
a nejprve expanduje "\next" pomocí "\expandafter" 
(tehdy si ještě pamatuje obsah "\next") a
vzápětí poté ukončí skupinu, vše tím zapomene 
a provede výsledek připravené expanze. Dílo je dokonáno.

Vraťme se k makru "\pdfunidefB". Toto makro vyhodnotí především kategorii
odloupnutého tokenu. Je-li to mezera, znak "^" nebo "_", přidá do "\out"
jejich odpovídající kód. Je-li to písmeno nebo ostatní znak, provede
"\pdfunidefC". Podívejme se tedy, co dělá "\pdfunidefC".

\begtt
\def\pdfunidefC {\edef\tmp{\expandafter\gobbletwowords\meaning\next}%
   \edef\out{\out\pdfunidefD}%
}
\endtt

Makro "\pdfunidefC" potřebuje zpětně zjistit, jaký znak je v "\next" uložen. K tomu
použije příkaz "\meaning\next", který vypíše třeba "the character B".
Pomocí "\gobbletwowords" odstraníme dvě slova "the character" a v makru "\tmp" zbude
jen "B". Do "\out" pak přidáme výsledek expanze "\pdfunidefD".

\begtt
\def\pdfunidefD{%
   \ifnum\sfcode\expandafter`\tmp=0 \Bslash 000\tmp
   \else
      \expandafter \insertbslashes \ifcase \sfcode\expandafter`\tmp\pdfunicodes 
         \else 000077\fi
   \fi
}
\endtt

Toto makro musí pracovat jen s expandovatelnými primitivy. Nejprve
"\pdfunidefD" zkontroluje, zda konvertovaný znak (skrytý v "\tmp") má sfkód
roven nula. V takovém případě expanduje na "\000<znak>". Jinak pomocí
"\expandafter" rozvine "\ifcase\sfcode`<znak>". Protože tato podmínka není
terminovaná, expanduje se i tělo tohoto "\ifcase" skryté v makru
"\pdfunicodes". Víme, že tam je psáno 

\begtt
\or<5ciferný kód>\or<5ciferný kód>\or<5ciferný kód>...
\endtt
Celý "\ifcase" je uzavřen pomocí "\else 000077\fi", což je kód otazníku.
Expandafter před "\ifcase" způsobí, že toto "\ifcase" expanduje na svou
odpovídající větev podle sfkódu, kde je "<5ciferný kód>". Před tímto
výsledkem se ovšem rozprostírá makro "\insertbslashes", které do tohoto
"<5ciferného kódu>" doplní první nulu a backslashe.

Zajímavý trik je na řádku těsně nad řadou "\fi\fi\fi\fi\fi\fi" v těle makra
"\pdfunidefB". Je tam řečeno "\expandafter\ifx\expandafter&\next".
Představme si, že "\next" je makro definované pomocí "\odef". V takovém
případě po vykonání těch dvou "\expandafter" dostáváme:

\begtt
\ifx&&<činnost>
\endtt
a provede se tedy požadovaná činnost. Je-li "\next" cokoli jiného, není to
shodné se znakem "&" a neprovede se nic. Je-li "\next" přímo znak "&",
neprovede se také nic.

Soubor "pdfuni.tex" je ukončen definicí makra z OPmac, která mění konvertor
spuštěný při činnosti "\outlines<num>" z původní hodnoty (konverze do ASCII) na novou
hodnotu (konverze pomocí "\pdfunidef").

\begtt
\def\cnvhook #1#2{#2\pdfunidef\tmp\tmp} % the default convertor in OPmac is
                                        % redefined here
\endtt

\sec Reference

\bib [pdfstandard]
\url{http://www.adobe.com/devnet/acrobat/pdfs/pdf_reference_1-7.pdf}

\bib [opmac] \url{http://petr.olsak.net/opmac.html}

\end
 
