%
% This file presents the 'verbose-ibid' style
%
\documentclass[a4paper]{article}
\usepackage[T1]{fontenc}
\usepackage[utf8]{inputenc}
\usepackage[american]{babel}
\usepackage{csquotes}
\usepackage[style=verbose-ibid,backend=biber]{biblatex}
\usepackage{hyperref}
\addbibresource{biblatex-examples.bib}
\newcommand{\cmd}[1]{\texttt{\textbackslash #1}}
\begin{document}

\section*{The \texttt{verbose-ibid} style}

This citation style is a slightly more compact variant of the
\texttt{verbose} style. Immediately repeated citations are replaced
by the abbreviation `ibidem' unless the citation is the first one on
the current page or double page spread (depending on the setting of
the \texttt{pagetracker} package option). This style is also
intended for citations given in footnotes.

\subsection*{Additional package options}

\subsubsection*{The \texttt{ibidpage} option}

The scholarly abbreviation \emph{ibidem} is sometimes taken to mean
both `same author~+ same title' and `same author~+ same title~+ same
page' in traditional citation schemes. By default, this is not the
case with this style because it may lead to ambiguous citations.
With \texttt{ibidpage=true} a page range postnote will be suppressed
in an \emph{ibidem} citation if the last citation was to the same
page range. With \texttt{ibidpage=false} the postnote is not omitted.
Citations to different page ranges than the previous always produce
the page ranges with either setting.
The default setting is \texttt{ibidpage=false}.

Consider the following example citations
\begin{verbatim}
\cite[12]{cicero}
\cite[12]{cicero}
\cite[12]{worman}
\cite[13]{worman}
\end{verbatim}
%
If \texttt{ibidpage} is set to \texttt{true}, the citations
come out -- shortened -- as
\begin{quote}
Cicero, \emph{De natura deorum,} p.~12

ibid.

Worman, \emph{The Cast of Character,} p.~12

ibid., p.~13
\end{quote}
%
The shortened result for \texttt{ibidpage=false} is
\begin{quote}
Cicero, \emph{De natura deorum,} p.~12

ibid., p.~12

Worman, \emph{The Cast of Character,} p.~12

ibid., p.~13
\end{quote}

\subsubsection*{The \texttt{dashed} option}

Use this option to fine-tune the formatting of the \texttt{pages}
and \texttt{pagetotal} fields in verbose citations. When an entry
with a \texttt{pages} field is cited for the first time and the
\texttt{postnote} is a page number or a page range, the citation
will end with two page specifications:

\begin{quote}
Author. \enquote{Title.} In: \emph{Book,} pp.\,100--150, p.\,125.
\end{quote}
%
In this example, \enquote{125} is the \texttt{postnote} and
\enquote{100--150} is the \texttt{pages} field (there are similar
issues with the \texttt{pagetotal} field). This may be confusing to
the reader. The \texttt{citepages} option controls how to deal with
these fields in this case. The option works as follows, given these
citations as an example:

\begin{verbatim}
\cite{key}
\cite[a note]{key}
\cite[125]{key}
\end{verbatim}
%
\texttt{citepages=permit} allows duplicates, i.e., the style will
print both the \texttt{pages}\slash \texttt{pagetotal} and the
\texttt{postnote}. This is the default setting:

\begin{quote}
Author. \enquote{Title.} In: \emph{Book,} pp.\,100--150.

Author. \enquote{Title.} In: \emph{Book,} pp.\,100--150, a note.

Author. \enquote{Title.} In: \emph{Book,} pp.\,100--150, p.\,125.
\end{quote}
%
\texttt{citepages=suppress} unconditionally suppresses the
\texttt{pages}\slash \texttt{pagetotal} fields in citations,
regardless of the \texttt{postnote}:

\begin{quote}
Author. \enquote{Title.} In: \emph{Book.}

Author. \enquote{Title.} In: \emph{Book,} a note.

Author. \enquote{Title.} In: \emph{Book,} p.\,125.
\end{quote}
%
\texttt{citepages=omit} suppresses the \texttt{pages}\slash
\texttt{pagetotal} in the third case only. They are still printed if
there is no \texttt{postnote} or if the \texttt{postnote} is not a
number or range:

\begin{quote}
Author. \enquote{Title.} In: \emph{Book,} pp.\,100--150.

Author. \enquote{Title.} In: \emph{Book,} pp.\,100--150, a note.

Author. \enquote{Title.} In: \emph{Book,} p.\,125.
\end{quote}
%
\texttt{citepages=separate} separates the \texttt{pages}\slash
\texttt{pagetotal} from the \texttt{postnote} in the third case:

\begin{quote}
Author. \enquote{Title.} In: \emph{Book,} pp.\,100--150.

Author. \enquote{Title.} In: \emph{Book,} pp.\,100--150, a note.

Author. \enquote{Title.} In: \emph{Book,} pp.\,100--150, esp. p.\,125.
\end{quote}
%
The string \enquote{especially} in the third case is the bibliography
string \texttt{thiscite}, which may be redefined.

\subsubsection*{The \texttt{dashed} option}

By default, this style replaces recurrent authors/editors in the
bibliography by a dash so that items by the same author or editor
are visually grouped. This feature is controlled by the package
option \texttt{dashed}. Setting \texttt{dashed=false} in the
preamble will disable this feature. The default setting is
\texttt{dashed=true}.

\subsection*{Hints}

If you want terms such as \emph{ibidem} to be printed in italics,
redefine \cmd{mkibid} as follows:

\begin{verbatim}
\renewcommand*{\mkibid}{\emph}
\end{verbatim}

\clearpage

\subsection*{\cmd{footcite} examples}

% The initial citation of an entry includes all the data.
This is just filler text.\footcite{aristotle:anima}
This is just filler text.\footcite{aristotle:physics}
% Subsequent citations use a more compact format.
This is just filler text.\footcite{aristotle:anima}
This is just filler text.\footcite{aristotle:physics}
% Immediately repeated citations are replaced by the
% abbreviation `ibidem'...
This is just filler text.\footcite{aristotle:physics}
\clearpage
% ... unless the citation is the first one on the current page
% or double page spread (depending on the setting of the
% `pagetracker' package option).
This is just filler text.\footcite{aristotle:physics}
This is just filler text.\footcite{aristotle:physics}

\clearpage

% If the `shorthand' field is defined, the shorthand is introduced
% on the first citation.
This is just filler text.\footcite{kant:kpv}
This is just filler text.\footcite{kant:ku}
% All subsequent citations will then use the shorthand.
This is just filler text.\footcite[24]{kant:kpv}
This is just filler text.\footcite[59--63]{kant:ku}

\clearpage

\subsection*{\cmd{autocite} examples}

% The \autocite command works like \footcite. Note that
% the period is moved and placed before the footnote.

This is just filler text \autocite{aristotle:rhetoric}.
This is just filler text \autocite{averroes/bland}.
This is just filler text \autocite{aristotle:rhetoric}.
This is just filler text \autocite{aristotle:anima}.
This is just filler text \autocite{aristotle:physics}.
This is just filler text \autocite{aristotle:physics}.

\clearpage

% Since all bibliographic data is provided on the first citation,
% this style may be used without a list of references and
% shorthands. Of course these lists may still be printed if desired.

\printshorthands
\printbibliography

\end{document}
